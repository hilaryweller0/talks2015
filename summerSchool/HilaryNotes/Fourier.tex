
\chapter{Fourier Analysis}

\liststepwise{

Fourier analysis and Fourier series are used for:\step{
\begin{itemize}
\item Analysing data (eg climate data)\step{
\item Numerical methods}\step{
\item Numerical analysis of methods}
\end{itemize}
}\step{


\section{Fourier Series}

Any periodic, integrable function, $f(x)$ (defined on $[-\pi,\pi]$),
can be expressed as a Fourier series; an infinite sum of sines and
cosines: 
\begin{equation}
f(x)=\frac{a_{0}}{2}+\sum_{k=1}^{\infty}a_{k}\cos kx+\sum_{k=1}^{\infty}b_{k}\sin kx\label{eqn:FourierSinCos}
\end{equation}


}\step{

\begin{minipage}[t]{0.6\columnwidth}%
The first three Fourier modes of a noisy function.\\
 \includegraphics[width=0.8\linewidth]{\lyxdot \lyxdot /\lyxdot \lyxdot /\lyxdot \lyxdot /\lyxdot \lyxdot /teaching/MTMW12/2014/notes/figs/smoothFunc}%
\end{minipage}}\step{\hfill{}%
\begin{minipage}[t]{0.38\columnwidth}%
\begin{itemize}
\item The $a_{k}$ and $b_{k}$ are the Fourier coefficients and the sines
and cosines are the Fourier modes. \step{
\item The more Fourier modes that included, the closer their sum will get
to the original function. }\end{itemize}
%
\end{minipage}

}}\clearpage{}

\begin{minipage}[c]{0.5\columnwidth}%
\noindent \begin{flushleft}
The first four Fourier modes of a square wave. The additional oscillations
are ``\textit{spectral ringing}'' 
\par\end{flushleft}%
\end{minipage}\hfill{}%
\begin{minipage}[c]{0.45\linewidth}%
\includegraphics[width=1\linewidth]{\lyxdot \lyxdot /\lyxdot \lyxdot /\lyxdot \lyxdot /\lyxdot \lyxdot /teaching/MTMW12/2014/notes/figs/squareWave-Fourier_Series}%
\end{minipage}\clearpage{}

Equivalently, equation (\ref{eqn:FourierSinCos}) can be expressed
as an infinite sum of exponentials: 
\begin{equation}
f(x)=\frac{a_{0}}{2}+\sum_{k=1}^{\infty}a_{k}\cos kx+\sum_{k=1}^{\infty}b_{k}\sin kx=\sum_{k=-\infty}^{\infty}A_{k}e^{ikx}.\label{eqn:FourierExp}
\end{equation}



\subsubsection*{Exercise}

Evaluate the $A_{k}$s in terms of the $a_{k}$s and $b_{k}$s.

\optparagraph{For $k=0$, $A_{0}=a_{0}/2$\\
For $k\ne0$, substitute $e^{ikx}=\cos kx+i\sin kx$ into eqn (\ref{eqn:FourierExp})
and consider one value of $k$:\\
 $a_{k}\cos kx+b_{k}\sin kx=A_{k}\left(\cos kx+i\sin kx\right)+A_{-k}\left(\cos kx-i\sin kx\right)$.\\
Assume $A_{k}=c+id$ and $A_{-k}=e+if$ where $c,d,e,f\in\mathbb{R}$.
Substituting in gives\\
\begin{eqnarray*}
a_{k}\cos kx+b_{k}\sin kx & = & \left(c+id\right)\left(\cos kx+i\sin kx\right)+\left(e+if\right)\left(\cos kx-i\sin kx\right)\\
 & = & \left(c+e\right)\cos kx+\left(f-d\right)\sin kx+i\left(\left(d+f\right)\cos kx+\left(c-e\right)\sin kx\right)
\end{eqnarray*}
Equating coefficients of $\cos kx\ $, $\sin kx\ $, $i\cos kx\ $
and $i\sin kx\ $ gives\\
$a_{k}=c+e\ $, $b_{k}=f-d\ $, $0=d+f\ $, $0=c-e$ $\implies d=-f,\ c=e,\ a_{k}=2c$,
$b_{k}=-2d$\\
$\implies A_{k}=\half\left(a_{k}-ib_{k}\right)\ $, $A_{-k}=\half\left(a_{k}+ib_{k}\right)$}


\subsubsection*{\clearpage{}Some Animations from Wikipedia}

First 5 Fourier modes of a saw tooth wave

\begin{minipage}[t]{1\columnwidth}%
\mediaMovie[autostart,loop]
{\includegraphics[width=\linewidth]{../../../../teaching/MTMW12/2014/notes/anims/Fourier/Periodic_identity_function/anim4.pdf}}
{animations/teaching_MTMW12_2014_notes_anims_Fourier_Periodic_identity_function.mp4}%
\end{minipage}

\begin{minipage}[t]{0.47\columnwidth}%
First 4 Fourier modes of the square waves

\mediaMovie[autostart,loop]{\includegraphics[width=\linewidth]{../../../../teaching/MTMW12/2014/notes/anims/Fourier/Fourier_series_square_wave_circles_animation/anim59.pdf}}
{animations/teaching_MTMW12_2014_notes_anims_Fourier_Fourier_series_square_wave_circles_animation.mp4}%
\end{minipage}\hfill{}%
\begin{minipage}[t]{0.47\columnwidth}%
First 4 Fourier modes of a saw tooth wave

\mediaMovie[autostart,loop]{\includegraphics[width=\linewidth]{../../../../teaching/MTMW12/2014/notes/anims/Fourier/Fourier_series_sawtooth_wave_circles_animation/anim59.pdf}}
{animations/teaching_MTMW12_2014_notes_anims_Fourier_Fourier_series_sawtooth_wave_circles_animation.mp4}%
\end{minipage}\clearpage{}


\section{Fourier Transform}

\stepwise*{\step{The Fourier Transform transforms a function $f$
which is defined over space (or time) into the frequency domain, so
that it is defined in terms of Fourier coefficients. }\step{ The
Fourier transform calculates the Fourier coefficients as: 
\[
a_{k}=\frac{1}{\pi}\int_{-\pi}^{\pi}f(x)\cos(kx)dx~,~~b_{k}=\frac{1}{\pi}\int_{-\pi}^{\pi}f(x)\sin(kx)dx
\]
}\step{


\section{Discrete Fourier Transform}

}\step{A discrete Fourier Transform converts a list of $2N+1$ equally
spaced samples of a real valued, periodic function, $f_{n}$, to the
list of the first $N+1$ complex valued Fourier coefficients:}\step{
\[
A_{k}=\frac{1}{N}\sum_{n=-N}^{N}f_{n}\ e^{-i\pi knx/N}.
\]
}\step{The truncated Fourier series:
\[
f\left(x\right)\approx\sum_{k=-N}^{N}A_{k}e^{ikx}
\]
is an approximation to the function $f$ which fits the sampled points,
$f_{n}$, exactly. }\step{

On a computer this is done with a \textbf{Fast Fourier Transform}
(or \texttt{fft}). The inverse Fourier transform (sometimes called
\texttt{ifft}) transforms the Fourier coefficients back to the $f$
values (transforming from spectral back to real space):

\begin{tabular}{ccc}
$f_{0},f_{1},f_{2},\cdots f_{2N}$  & $\xrightarrow{\text{{\tt fft}}}$  & $A_{0},A_{1},\cdots A_{N}$ \tabularnewline
$A_{0},A_{1},\cdots A_{N}$  & $\xrightarrow{\text{{\tt ifft}}}$  & $f_{0},f_{1},f_{2},\cdots f_{2N}$ \tabularnewline
\end{tabular}}}

\clearpage{}


\section{Differentiation and Interpolation}

If we know the Fourier coefficients, $A_{k}$, of a function $f$
then we can calculate the gradient of $f$ at any point, $x$: If
\begin{equation}
f(x)=\sum_{k=0}^{\infty}a_{k}\cos kx+\sum_{k=0}^{\infty}b_{k}\sin kx=\sum_{k=-\infty}^{\infty}A_{k}e^{ikx}
\end{equation}
then\stepwise{ 
\begin{equation}
f^{\prime}(x)=\opttext{\sum_{k=0}^{\infty}-ka_{k}\sin kx+\sum_{k=0}^{\infty}kb_{k}\cos kx=\sum_{k=-\infty}^{\infty}i~k~A_{k}e^{ikx}.}\label{eqn:FourierGrad}
\end{equation}
and the second derivative: 
\begin{equation}
f^{\prime\prime}(x)=\opttext{\sum_{k=0}^{\infty}-k^{2}a_{k}\cos kx-\sum_{k=0}^{\infty}k^{2}b_{k}\sin kx=\sum_{k=-\infty}^{\infty}-k^{2}~A_{k}e^{ikx}.}\label{eqn:FourierGrad2}
\end{equation}
\step{These have spectral accuracy; the order of accuracy is as high
as the number of points. Similarly equation \ref{eqn:FourierSinCos}
or \ref{eqn:FourierExp} can be used directly to interpolate $f$
onto an undefined point, $x$. Again, the order of accuracy is spectral.}\step{


\section{Spectral Models}
\begin{itemize}
\item ECMWF use a spectral model. 
\item The prognostic variables are transformed between physical and spectral
space using \texttt{fft}s and \texttt{ifft}s. 
\item Gradients are calculated very accurately in spectral space
\end{itemize}
}}


\section{Wave Power and Frequency}

\begin{minipage}[c]{0.55\linewidth}%
 \setlength{\parskip}{6pt} \setlength{\parindent}{0pt} $\sin kx$
and $\cos kx$ are waves:

\setlength{\tabcolsep}{4pt} %
\begin{tabular}{lcl}
wavenumber  & $k$  & Number of complete waves \tabularnewline
(or frequency)  &  & that will fit into the interval \tabularnewline
 &  & $[-\pi,\pi]$\tabularnewline
wavelength  &  & $\lambda=\frac{2\pi}{k}$ \tabularnewline
\end{tabular}%
\end{minipage}%
\begin{minipage}[c]{0.43\linewidth}%
 \includegraphics[width=1\linewidth]{\lyxdot \lyxdot /\lyxdot \lyxdot /\lyxdot \lyxdot /\lyxdot \lyxdot /teaching/MTMW12/2014/notes/figs/sineWaves} %
\end{minipage}
\begin{itemize}
\item If a function, $f$, has Fourier coefficients, $a_{k}$ and $b_{k}$,
then wavenumber $k$ has power $a_{k}^{2}+b_{k}^{2}$.
\item A plot of wave frequency versus power is referred to as the power
spectrum. 
\end{itemize}
\clearpage{}


\section{Recap Questions}

\stepwise{
\begin{enumerate}
\item In the Fourier decomposition
\[
f(x)=\frac{a_{0}}{2}+\sum_{k=1}^{\infty}a_{k}\cos kx+\sum_{k=1}^{\infty}b_{k}\sin kx
\]
what are:

\begin{enumerate}
\item the Fourier coefficients \hfill{}\opttext{(the $a_k$ and the $b_k$)}
\item the Fourier modes \hfill{}\opttext{(the sines and cosines)}
\item the wavenumbers (or frequencies) \hfill{}\opttext{(the $k$s)}
\item the power of wavenumber $k$ \hfill{}\opttext{($a_k^2+b_k^2$)}
\end{enumerate}
\item How would you describe the operation:
\[
a_{k}=\frac{1}{\pi}\int_{-\pi}^{\pi}f(x)\cos(kx)dx~,~~b_{k}=\frac{1}{\pi}\int_{-\pi}^{\pi}f(x)\sin(kx)dx
\]
\phantom{}\hfill{}\opttext{(a Fourier transform)}
\item Given a list of $2N+1$ equally spaced samples of a real valued, periodic
function, $f_{n}$, how would you describe the following operation
to convert this into a a list of $N+1$ values:
\[
A_{k}=\frac{1}{N}\sum_{n=-N}^{N}f_{n}\ e^{-i\pi knx/N}
\]
\hfill{}\opttext{(a discrete Fourier transform)}
\item What is the wavelength of a wave described by $\sin4x$ \hfill{}\opttext{($2\pi/4$)}
\end{enumerate}
}

\clearpage{}


\section{Analysing Power Spectra}

\liststepwise{

\begin{minipage}[c]{0.7\linewidth}%
\begin{center}
\textbf{Daily rainfall at a station in the Middle East for 21 years}\\
\textbf{ \includegraphics[width=1\linewidth]{\lyxdot \lyxdot /\lyxdot \lyxdot /\lyxdot \lyxdot /\lyxdot \lyxdot /teaching/MTMW12/2014/notes/figs/rainSmoothed} }
\par\end{center}%
\end{minipage}\hfill{}%
\begin{minipage}[c]{0.29\linewidth}%
\begin{flushleft}
\textbf{Truncated Fourier filtered rainfall: }
\par\end{flushleft}
\begin{itemize}
\item \begin{flushleft}
\opttext{very smooth (only low wavenumbers included)}
\par\end{flushleft}
\item \opttext{includes negative values -- \textit{``spectral ringing''}}\end{itemize}
%
\end{minipage}

\step{%
\begin{minipage}[c]{0.6\linewidth}%
\begin{center}
\textbf{Power Spectrum of Middle East Rainfall \includegraphics[width=1\linewidth]{\lyxdot \lyxdot /\lyxdot \lyxdot /\lyxdot \lyxdot /\lyxdot \lyxdot /teaching/MTMW12/2014/notes/figs/rainPower} }
\par\end{center}%
\end{minipage}\hfill{}%
\begin{minipage}[c]{0.38\linewidth}%
\textbf{Observations}
\begin{itemize}
\item \begin{flushleft}
\opttext{dominant frequency at one year (annual cycle)}
\par\end{flushleft}
\item \begin{flushleft}
\opttext{power at high frequencies (ie daily variability)}
\par\end{flushleft}\end{itemize}
%
\end{minipage}

number per year = wavenumber$\times$365/total number of days}}\clearpage{}


\subsection*{Time Series of the Nino 3 sea surface temperature (SST)}

\liststepwise{The SST in the Nino 3 region of the equatorial Pacific
is a diagnostic of El Nino

\begin{minipage}[c]{0.7\linewidth}%
 \includegraphics[width=1\linewidth]{\lyxdot \lyxdot /\lyxdot \lyxdot /\lyxdot \lyxdot /\lyxdot \lyxdot /teaching/MTMW12/2014/notes/figs/n3FourierModes} %
\end{minipage}\hfill{}%
\begin{minipage}[c]{0.29\linewidth}%
\begin{itemize}
\item \begin{flushleft}
\opttext{Annual cycle is the Fourier mode at 1 year }
\par\end{flushleft}
\item \begin{flushleft}
\opttext{The ``two years and slower'' filtered data is the sum
of all the Fourier modes of these frequencies.} 
\par\end{flushleft}\end{itemize}
%
\end{minipage}\step{

\begin{minipage}[c]{0.5\linewidth}%
\begin{center}
\textbf{Power Spectrum of Nino 3 SST}\\
\textbf{\includegraphics[width=1\linewidth]{\lyxdot \lyxdot /\lyxdot \lyxdot /\lyxdot \lyxdot /\lyxdot \lyxdot /teaching/MTMW12/2014/notes/figs/n3Power}}
\par\end{center}%
\end{minipage}\hfill{}%
\begin{minipage}[c]{0.48\linewidth}%
\begin{itemize}
\item \begin{flushleft}
\opttext{Dominant frequency at 1 year (annual cycle)}
\par\end{flushleft}
\item \begin{flushleft}
\opttext{Less power at high frequencies (SST varies slowly)}
\par\end{flushleft}
\item \begin{flushleft}
\opttext{Power at 1-10 years (El Nino every 3-7 years)}
\par\end{flushleft}\end{itemize}
%
\end{minipage}}}

\clearpage{}


\subsection*{Time Series of the Quasi-Biennial Oscillation (QBO)}

\liststepwise{The QBO is an oscillation of the equatorial zonal wind
between easterlies and westerlies in the tropical stratosphere which
has a mean period of 28 to 29 months:

\includegraphics[width=1\linewidth]{/home/hilary/latex/teaching/MTMW12/2011Notes/practicals/3_Fourier/moreData/qbo}

\step{

\begin{minipage}[c]{0.65\linewidth}%
\begin{center}
\textbf{Power Spectrum of QBO}\\
\includegraphics[width=1\linewidth]{/home/hilary/latex/teaching/MTMW12/2011Notes/practicals/3_Fourier/moreData/qboPower}
\par\end{center}%
\end{minipage}\hfill{}%
\begin{minipage}[c]{0.33\linewidth}%
\begin{itemize}
\item \begin{flushleft}
\opttext{Dominant frequency at close to 2 years}
\par\end{flushleft}
\item \begin{flushleft}
\opttext{Less power at high frequencies}
\par\end{flushleft}
\item \begin{flushleft}
\opttext{Less power at long time-scales}
\par\end{flushleft}\end{itemize}
%
\end{minipage}}

\step{Knowledge of Fourier analysis is necessary for analysing stability
of numerical methods ...}}
