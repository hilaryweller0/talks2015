
\chapter{Modelling Wave Equations}

\liststepwise*{\step{

Many of the processes in the atmosphere are represented by the shallow
water equations (SWE). The assumptions needed to derive the SWE are:
\begin{itemize}
\item Horizontal length scale $>>$ vertical length scale
\item Very small vertical velocities
\end{itemize}
Depth integrate the Navier-Stokes equations over orography to give
the SWE:}\step{

\begin{minipage}[c]{0.54\linewidth}%
\begin{align}
\frac{D\mathbf{u}}{Dt} & =-2\Omega\times\mathbf{u}-g\nabla(h+h_{0})+\mu_{u}\nabla^{2}\mathbf{u}\label{eqn:SWEuv}\\
\frac{Dh}{Dt} & +h\nabla\cdot\mathbf{u}=0\label{eqn:SWEh}
\end{align}
%
\end{minipage}\hfill{}%
\begin{minipage}[c]{0.4\linewidth}%
 \resizebox{1\columnwidth}{!}{\input{figs/shallowWater.pdftex_t}} %
\end{minipage}

where \global\long\def\arraystretch{1}


\noindent \begin{flushleft}
\begin{tabular}{llll}
$\mathbf{u}$  & Depth intetraged wind vector  & $g$  & Acceleration due to gravity\tabularnewline
$t$  & Time  & $\nabla$ & Gradients in the horizontal\tabularnewline
$\Omega$  & Rotation rate of planet & $h_{0}$  & Height of the bottom topography\tabularnewline
$h$  & Fluid depth & $\mu_{u}$  & Diffusion of momentum\tabularnewline
\end{tabular}
\par\end{flushleft}

}\step{

\textbf{Exercise: }Considering the meaning of the terms in the momentum
equation in section \ref{sec:momEqn}, what are the meaning of the
terms of the momentum equation of the SWE?

}}

\clearpage{}


\section{Processes Represented by the SWE}

\liststepwise{

Which of these processes are represented by the SWE and which are
only represented by the full NS equations?

\begin{tabular}{llll}
Horizontal advection & \opttext{SWE} & Acoustic waves & \opttext{NS}\tabularnewline
Vertical advection & \opttext{NS} & Coriolis & \opttext{SWE}\tabularnewline
Gravity waves & \opttext{SWE} & Diffusion & \opttext{SWE}\tabularnewline
Rossby waves & \opttext{SWE} & Heat transport & \opttext{NS}\tabularnewline
Adiabatic expansion & \opttext{NS} & Atmospheric convection & \opttext{NS}\tabularnewline
Geostrophic balance & \opttext{SWE} & Geostrophic turbulence & \opttext{SWE}\tabularnewline
\end{tabular}

\step{\vfill{}



\subsection*{Component Form of the SWE}

Assuming $\mathbf{u}=(u,v,0)^{T}$ and $2\Omega=(0,0,f)^{T}$, equations
(\ref{eqn:SWEuv}) and (\ref{eqn:SWEh}) written in component form
are:
\begin{eqnarray*}
\frac{\partial u}{\partial t}+u\frac{\partial u}{\partial x}+v\frac{\partial u}{\partial y} & = & \phantom{-}fv-g\frac{\partial(h+h_{0})}{\partial x}+\mu_{u}\left(\frac{\partial^{2}u}{\partial x^{2}}+\frac{\partial^{2}u}{\partial y^{2}}\right)\\
\frac{\partial v}{\partial t}+u\frac{\partial v}{\partial x}+v\frac{\partial v}{\partial y} & = & -fu-g\frac{\partial(h+h_{0})}{\partial y}+\mu_{u}\left(\frac{\partial^{2}v}{\partial x^{2}}+\frac{\partial^{2}v}{\partial y^{2}}\right)\\
\frac{\partial h}{\partial t}+u\frac{\partial h}{\partial x}+v\frac{\partial h}{\partial y}+h\left(\frac{\partial u}{\partial x}+\frac{\partial v}{\partial y}\right)=0 & \text{or} & \frac{\partial h}{\partial t}+\frac{\partial hu}{\partial x}+\frac{\partial hv}{\partial y}=0
\end{eqnarray*}


}}

\clearpage{}


\subsection{Linearised SWE}

\liststepwise*{\step{

In order to find analytic solutions and to analyse the behaviour of
numerical methods, we linearise the SWE. }\step{ Assume:
\begin{itemize}
\item $\mathbf{u}=(u,v,0)^{T}$ is small
\item 2$\Omega=(0,0,f)^{T}$
\item $h=H+h^{\prime}$ where $H$ is uniform in space and time and $h^{\prime}$
is small
\item the product of two small variables is ignored (including if one or
both are inside a differential)
\item $h_{0}$ and $\mu_{u}$ are ignored
\end{itemize}
This gives the following equations for $u,$$v$ and $h^{\prime}$
expressed in terms of $f$ (rather than $\Omega$):}\opttext{
\begin{eqnarray}
\frac{\partial u}{\partial t} & = & \phantom{-}fv-g\frac{\partial h^{\prime}}{\partial x}\label{eq:linSWEu}\\
\frac{\partial v}{\partial t} & = & -fu-g\frac{\partial h^{\prime}}{\partial y}\label{eq:linSWEv}\\
\frac{\partial h^{\prime}}{\partial t} & = & -H\left(\frac{\partial u}{\partial x}+\frac{\partial v}{\partial y}\right)\label{eq:linSWEh}
\end{eqnarray}


}}

\clearpage{}

\liststepwise*{\step{


\section{Analytic Solultion}

Ignoring Coriolis, the linearised SWE have analytic solutions. In
1d these are: 
\begin{align}
h^{\prime} & =\hphantom{\pm\sqrt{g/H}\ }\mathbb{H}\ e^{ikx}\ e^{\pm ikt\sqrt{gH}}\label{eq:hwaves}\\
u & =\pm\sqrt{g/H}\ \mathbb{H}\ e^{ikx}\ e^{\pm ikt\sqrt{gH}}\label{eq:uwaves}
\end{align}
for any constant $\mathbb{H}$. So waves with wavenumber $k$ in space
oscillate with frequency $k\sqrt{gH}$ in time and the gravity speed
is ... $\opttext{\sqrt{gH}}$ for waves of all wavelengths (the gravity
waves are non-dispersive since the wave speed does not depend on the
wavelength).}\step{


\section{Unstaggered Forward-Backward (1d A-grid FB)\label{sec:FB_1dA}}

As there are two equations that depend on each other, it is quite
natural to solve them using forward-backward time-stepping. We will
also start by assuming that $h$ and $u$ are defined at the same
spatial positions (this is called co-located, unstaggered or A-grid)
and we will use centred spatial discretisation:}\step{ 
\begin{align}
\frac{\partial u}{\partial t} & =-g\frac{\partial h}{\partial x}~\rightarrow & \opttext{\frac{u_{j}^{(n+1)}-u_{j}^{(n)}}{\Delta t}}\ \ \  & =\opttext{-g\frac{h_{j+1}^{(n)}-h_{j-1}^{(n)}}{2\Delta x}}\ \ \ \ \ \ \ \ \ \ \ \ \ \ \ \ \ \ \label{eqn:SWEuFBA}\\
\frac{\partial h}{\partial t} & =-H\frac{\partial u}{\partial x}~\rightarrow & \opttext{\frac{h_{j}^{(n+1)}-h_{j}^{(n)}}{\Delta t}}\ \ \  & =\opttext{-H\frac{u_{j+1}^{(n+1)}-u_{j-1}^{(n+1)}}{2\Delta x}}\ \ \ \ \ \ \ \ \ \ \ \ \ \ \ \ \ \ \label{eqn:SWEhFBA}
\end{align}
where $x_{j}=j\Delta x$, $t^{(n)}=n\Delta t$, $h_{j}^{(n)}=h(x_{j},t^{(n)})$
and $u_{j}^{(n)}=u(x_{j},t^{(n)})$.

}}\clearpage{}


\subsection{Von-Neumann Stability Analysis}

\liststepwise*{\step{

We can find the stability limits and wave frequency as a function
of wavenumber for the numerical scheme given in section \ref{sec:FB_1dA}
(1d A-grid FB) using von-Neumann stability analysis (details beyond
the scope of this course but are reproduced on the next slide). The
outcomes are that 1d A-grid FB is:}
\begin{itemize}
\item \step{Stable for Courant number $|c|\le2$ where $c=\sqrt{gH}\Delta t/\Delta x$}\step{
\item Wave frequency dependent on wavenumber like:
\[
\omega=\tan^{-1}\frac{\frac{c}{2}\sin k\Delta x\sqrt{4-c^{2}\sin^{2}k\Delta x}}{1-\frac{c^{2}}{2}\sin^{2}k\Delta x}
\]
}
\end{itemize}
\step{This is the A-grid dispersion relation:

\includegraphics[width=0.5\linewidth]{figs/A_dispersion}}\step{

Grid-scale gravity waves ($k\Delta x/\pi=1$) have zero frequency!
This is highly unrealistic.}}

\clearpage{}


\subsubsection*{Outline of the von-Neumann Stability Analysis of 1d A-grid FB}

To calculate an amplification factor, $A$, for each wavenumber, $k$,
we assume wave-like solutions for $h$ and $u$: 
\begin{align}
h_{j}^{(n)} & =\mathbb{H}~A^{n}~e^{ikj\Delta x}\\
u_{j}^{(n)} & =\mathbb{U}~A^{n}~e^{ikj\Delta x}
\end{align}
Substituting these into (\ref{eqn:SWEuFBA}) and (\ref{eqn:SWEhFBA})
leads to: 
\begin{equation}
A=1-\frac{c^{2}}{2}\sin^{2}k\Delta x\pm\frac{ic}{2}\sin k\Delta x\sqrt{4-c^{2}\sin^{2}k\Delta x}
\end{equation}
where $c=\frac{\sqrt{gH}\Delta t}{\Delta x}$. There are two solutions
for $A$ but this is correct because there are also two analytic solutions
to the equations (because of the $\pm$ in the analytic solution).
For $|c|\le2$ this gives $|A|^{2}=1$ so the scheme is stable and
undamping for sufficiently small time steps. However for $|c|>2$
we have: 
\[
|A|^{2}=\biggl(1-\frac{c^{2}}{2}\sin^{2}k\Delta x\pm\frac{c}{2}\sin k\Delta x\sqrt{c^{2}\sin^{2}k\Delta x-4}\biggr)^{2}
\]
which can be greater than 1 and so the scheme is unstable for $|c|>2$.

The argument $A$ gives us the wave frequency as a function of wavenumber:
\begin{equation}
\omega=\tan^{-1}\frac{\frac{c}{2}\sin k\Delta x\sqrt{4-c^{2}\sin^{2}k\Delta x}}{1-\frac{c^{2}}{2}\sin^{2}k\Delta x}
\end{equation}
This can be simplified by assuming that $\frac{c}{2}\sin k\Delta x=\sin\alpha$
to give: 
\begin{equation}
\omega=\pm2\alpha=\pm2\sin^{-1}\bigl(\frac{c}{2}\sin k\Delta x\bigr)
\end{equation}


\clearpage{}


\section{Problems with Co-location of $h$ and $u$}

\stepwise*{

Consider the following initial conditions of the linearised non-rotating
SWE:

\resizebox{0.48\linewidth}{!}{\input{figs/AgridMode.pdftex_t}} \hfill{}\resizebox{0.48\linewidth}{!}{\input{figs/AgridMode_u.pdftex_t}}

\textbf{Questions:}
\begin{enumerate}
\item \step{How do you expect the real solution of the linearised SWE to
evolve?}


\opttext{High-frequency waves will be generated that propagate in both directions. The solution will oscillate between having non-zero $h^\prime$ and non-zero $u$.}\vspace{2cm}
\step{

\item How will the solution of the 1d A-grid FB scheme evolve?}


\opttext{The solution will not change after initialisation. The grid-scale wave in $h^\prime$ will remain. No non-zero $u$ will be generated.}

\end{enumerate}
}

\clearpage{}


\section{Staggered Forward-Backward (1d C-grid FB)\label{sec:FB_1dC}}

\liststepwise{

So that gradients of $h$ can be calculated where $u$ is located
and gradients of $u$ can be calculated where $h$ is located, $h$
and $u$ can be staggered in space:\\
\hspace{3cm}\scalebox{0.75}[0.75]{\input{figs/C-grid.pdftex_t}}\\
Using centered, 2-point spatial differences and forward-backward in
time gives: 
\begin{align}
\frac{\partial u}{\partial t} & =-g\frac{\partial h}{\partial x}~~\rightarrow & \opttext{\frac{u_{j+\half}^{(n+1)}-u_{j+\half}^{(n)}}{\Delta t}}\ \ \  & =\opttext{-g\frac{h_{j+1}^{(n)}-h_{j}^{(n)}}{\Delta x}}\ \ \ \ \ \ \ \ \ \ \ \ \ \ \ \ \ \ \label{eqn:SWEuFBC}\\
\frac{\partial h}{\partial t} & =-H\frac{\partial u}{\partial x}~~\rightarrow & \opttext{\frac{h_{j}^{(n+1)}-h_{j}^{(n)}}{\Delta t}}\ \ \  & =\opttext{-H\frac{u_{j+\half}^{(n+1)}-u_{j-\half}^{(n+1)}}{\Delta x}}\ \ \ \ \ \ \ \ \ \ \ \ \ \ \ \ \ \ \label{eqn:SWEhFBC}
\end{align}
\step{Von-Neumann stability analysis gives:}
\begin{itemize}
\item \step{$|A|=\begin{cases}
1 & \text{for }|c|\le1\\
>1 & \text{for }|c|>1\text{ for some }k\Delta x
\end{cases}$ 


$\therefore$ neutrally stable for $|c|\le1$}\step{

\item Dispersion relation:\\
$\omega=\pm2\alpha=\pm2\sin^{-1}\bigl(c\sin\frac{k\Delta x}{2}\bigr)$\textbf{\hfill{}}\raisebox{-0.1\linewidth}[0pt][0pt]{\includegraphics[width=0.5\linewidth]{figs/AC_dispersion}}}\step{
\item $\therefore$ the C-grid is dispersive
\item grid-scale waves propagate too slowly}\step{
\item C-grid widely used in atmosphere and ocean models }\step{
\item What about in 2d?}
\end{itemize}
}

\clearpage{}


\section{Arakawa Grids}

\liststepwise{

In two dimensions, there are more possibilities for where the prognostic
variables are located:

\begin{tabular}{ccc}
A-grid & B-grid & C-grid \\
\ifthenelse{\boolean{@onlineversion}}%
{\switch{\resizebox{0.3\textwidth}{!}{\input{figs/Arakawa/grid.pdftex_t}}}
       {\resizebox{0.3\textwidth}{!}{\input{figs/Arakawa/A.pdf_t}}}}
{\ifthenelse{\boolean{@studentversion}}%
    {\resizebox{0.3\textwidth}{!}{\input{figs/Arakawa/grid.pdftex_t}}}
    {\resizebox{0.3\textwidth}{!}{\input{figs/Arakawa/A.pdf_t}}}}
&
\ifthenelse{\boolean{@onlineversion}}%
{\switch{\resizebox{0.3\textwidth}{!}{\input{figs/Arakawa/grid.pdftex_t}}}
       {\resizebox{0.3\textwidth}{!}{\input{figs/Arakawa/B.pdf_t}}}}
{\ifthenelse{\boolean{@studentversion}}%
    {\resizebox{0.3\textwidth}{!}{\input{figs/Arakawa/grid.pdftex_t}}}
    {\resizebox{0.3\textwidth}{!}{\input{figs/Arakawa/B.pdf_t}}}}
&
\ifthenelse{\boolean{@onlineversion}}%
{\switch{\resizebox{0.3\textwidth}{!}{\input{figs/Arakawa/grid.pdftex_t}}}
       {\resizebox{0.3\textwidth}{!}{\input{figs/Arakawa/C.pdf_t}}}}
{\ifthenelse{\boolean{@studentversion}}%
    {\resizebox{0.3\textwidth}{!}{\input{figs/Arakawa/grid.pdftex_t}}}
    {\resizebox{0.3\textwidth}{!}{\input{figs/Arakawa/C.pdf_t}}}}
\\
D-grid & E-grid &  \\
\ifthenelse{\boolean{@onlineversion}}%
{\switch{\resizebox{0.3\textwidth}{!}{\input{figs/Arakawa/grid.pdftex_t}}}
       {\resizebox{0.3\textwidth}{!}{\input{figs/Arakawa/D.pdf_t}}}}
{\ifthenelse{\boolean{@studentversion}}%
    {\resizebox{0.3\textwidth}{!}{\input{figs/Arakawa/grid.pdftex_t}}}
    {\resizebox{0.3\textwidth}{!}{\input{figs/Arakawa/D.pdf_t}}}}
&
\ifthenelse{\boolean{@onlineversion}}%
{\switch{\resizebox{0.3\textwidth}{!}{\input{figs/Arakawa/grid.pdftex_t}}}
       {\resizebox{0.3\textwidth}{!}{\input{figs/Arakawa/E.pdf_t}}}}
{\ifthenelse{\boolean{@studentversion}}%
    {\resizebox{0.3\textwidth}{!}{\input{figs/Arakawa/grid.pdftex_t}}}
    {\resizebox{0.3\textwidth}{!}{\input{figs/Arakawa/E.pdf_t}}}}
\end{tabular}

}\clearpage{}


\section{Linearised Shallow-Water Equations on Arakawa Grids}

The linearised SWE with rotation are:
\begin{align*}
\frac{\partial\mathbf{u}}{\partial t} & =-2\Omega\times\mathbf{u}-g\nabla h\\
\frac{\partial h^{\prime}}{\partial t} & +H\nabla\cdot(\mathbf{u})=0
\end{align*}

\begin{itemize}
\item The linearised SWE are solved numerically on Arakawa A, B and C grids,
starting from initial conditions consisting of zero velocity and zero
$h^{\prime}$ everywhere except a positive $h^{\prime}$ in one central
grid-box
\item The colours show $h^{\prime}$ in the grid boxes. Red positive, blue
negative, white zero
\end{itemize}
\stepwise{
\begin{tabular}{ccc}
A-grid & B-grid & C-grid \\
\switch{\includegraphics[width=0.3\linewidth]{2dSWE/Agrid1.pdf}}
{\movie*{10}{1}{\includegraphics[width=0.3\linewidth]{2dSWE/Agrid\thesubstep.pdf}}}
&
\switch{\includegraphics[width=0.3\linewidth]{2dSWE/Bgrid1.pdf}}
{\movie*{10}{1}{\includegraphics[width=0.3\linewidth]{2dSWE/Bgrid\thesubstep.pdf}}}
&
\switch{\includegraphics[width=0.3\linewidth]{2dSWE/Cgrid1.pdf}}
{\movie*{10}{1}{\includegraphics[width=0.3\linewidth]{2dSWE/Cgrid\thesubstep.pdf}}}
\end{tabular}
}
