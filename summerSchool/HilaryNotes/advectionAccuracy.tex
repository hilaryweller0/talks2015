\cleardoublepage{}


\section{Accuracy of Advection Schemes}

The 1d linear advection equation
\[
\frac{\partial\phi}{\partial t}+u\frac{\partial\phi}{\partial x}=0
\]



\subsection{Accuracy of FTBS}

FTBS for solving the linear advection equation (for constant $\Delta x$):
\[
\frac{\phi_{j}^{(n+1)}-\phi_{j}^{(n)}}{\Delta t}+u\frac{\phi_{j}^{(n)}-\phi_{j-1}^{(n)}}{\Delta x}=0
\]
Advection of an initial profile once around the domain (periodic boundaries,
$c=0.2$, 100 space intervals, 500 time steps):\\
{\mediaMovie[autostart,loop]
{\includegraphics[width=\textwidth]{/home/hilary/latex/teaching/MTMW12/2014/notes/anims/FTBS/FTBS_50.pdf}}
{animations/latex_teaching_MTMW12_2014_notes_anims_FTBS.mp4}}


\paragraph*{Questions}
\begin{itemize}
\item Why is it so inaccurate?
\item Why is it so diffusive?
\end{itemize}

\paragraph*{Answers}
\begin{itemize}
\item Use Taylor series to find the approximation for $\partial\phi/\partial x=\phi^{\prime}$
which is used in FTBS. 
\item What order accurate is it?
\item What is the entire leading error term?
\end{itemize}
\stepwise{

Write down the Taylor series approximation for $\phi_{j-1}$ about
$\phi_{j}$ and rearrange to find $\phi_{j}^{\prime}$:
\begin{align*}
\phi_{j-1} & \approx\opttext{\phi_{j}-\Delta x\phi_{j}^{\prime}+\frac{\Delta x^{2}}{2!}\phi_{j}^{\prime\prime}}\\
\Rightarrow\phi_{j}^{\prime} & =\opttext{\frac{\phi_{j}-\phi_{j-1}}{\Delta x}+\frac{\Delta x}{2}\phi_{j}^{\prime\prime}.}
\end{align*}
\vspace{2cm}

\begin{itemize}
\item \step{$\phi_{j}^{\prime\prime}$ is unknown so the leading error
term is $\frac{\Delta x}{2}\phi_{j}^{\prime\prime}$ so FTBS is first
order accurate.}
\item \step{The error is proportional to $\phi^{\prime\prime}=\partial^{2}\phi/\partial x^{2}$
which is the spatial term from the diffusion equation. Adding this
error to the advection equation makes the equation diffusive.}
\end{itemize}
}

\clearpage{}


\subsection{Accuracy of CTCS}

CTCS for solving the linear advection equation (for constant $\Delta x$):
\[
\frac{\phi_{j}^{(n+1)}-\phi_{j}^{(n-1)}}{2\Delta t}+u\frac{\phi_{j+}^{(n)}-\phi_{j-1}^{(n)}}{2\Delta x}=0
\]
Advection of an initial profile once around the domain (periodic boundaries,
$c=0.2$, 100 space intervals, 500 time steps):\\
\mediaMovie[autostart,loop]
{\includegraphics[width=\textwidth]{/home/hilary/latex/teaching/MTMW12/2014/notes/anims/FTBS_CTCS/FTBS_CTCS_50.pdf}}
{animations/latex_teaching_MTMW12_2014_notes_anims_FTBS_CTCS.mp4}


\paragraph*{Questions}
\begin{itemize}
\item Is CTCS more accurate than FTBS?
\item Why does it generate so many oscillations?
\end{itemize}

\paragraph*{Answers?}
\begin{itemize}
\item Use Taylor series to find the approximation for $\partial\phi/\partial x=\phi^{\prime}$
which is used in CTCS. 
\item What order accurate is it?
\item What is the entire leading error term?
\end{itemize}
\clearpage{}

Write down the Taylor series approximations for $\phi_{j-1}$ and
$\phi_{j+1}$ about $\phi_{j}$, eliminate $\phi_{j}^{\prime\prime}$
and rearrange to find $\phi_{j}^{\prime}$:\stepwise{

\begin{align*}
\phi_{j-1} & \approx\opttext{\phi_{j}-\Delta x\phi_{j}^{\prime}+\frac{\Delta x^{2}}{2!}\phi_{j}^{\prime\prime}-\frac{\Delta x^{3}}{3!}\phi_{j}^{\prime\prime\prime}}\\
\phi_{j+1} & \approx\opttext{\phi_{j}+\Delta x\phi_{j}^{\prime}+\frac{\Delta x^{2}}{2!}\phi_{j}^{\prime\prime}+\frac{\Delta x^{3}}{3!}\phi_{j}^{\prime\prime\prime}}\ \ \ \ \ \ \ \ \ \ \ \ \ \ \ \ \ \ \ \ \ \ \ \ \ \ \ \ \ \ \ \ \ \ \ \ \ \ \ \ \ \ \ \ \ \ \ \ \ \ \ .\\
\implies\phi_{j+1}-\phi_{j-1} & \approx\opttext{2\Delta x\phi_{j}^{\prime}+\frac{\Delta x^{3}}{3}\phi_{j}^{\prime\prime\prime}}\\
\implies\phi_{j}^{\prime} & \approx\opttext{\frac{\phi_{j+1}-\phi_{j-1}}{2\Delta x}-\frac{\Delta x^{2}}{3!}\phi_{j}^{\prime\prime\prime}}
\end{align*}
\vspace{0.1cm}

\begin{itemize}
\item \step{$\phi_{j}^{\prime\prime\prime}$ is unknown so the leading
error term is $\frac{\Delta x^{2}}{3!}\phi_{j}^{\prime\prime\prime}$
so CTCS is second order accurate.}
\item \step{The error is proportional to $\phi^{\prime\prime\prime}=\partial^{3}\phi/\partial x^{3}$
which is not diffusive.}
\item \step{So why the oscillations?}
\end{itemize}
}\pause


\subsubsection*{Godunov's theorem}

Linear numerical schemes for solving partial differential equations,
having the property of not generating new extrema (monotone scheme),
can be at most first-order accurate.\pause


\subsection*{Stability Analysis}

In order to predict the stability of different schemes, we will need
Fourier analysis ...
