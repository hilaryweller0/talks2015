
\chapter{Numerical Analysis of Advection Schemes \label{chap:advectionAnalysis}}


\section{Domain of Dependence}

If I stand outside on a windy day, what does the temperature depend
on?

\stepwise{\opttext{
\begin{itemize}
\item Solar radiation\opttext{
\item Wind direction - why?}\opttext{

\begin{itemize}
\item Temperature upwind
\end{itemize}
\item }\opttext{Altitude}
\end{itemize}
}\step{The domain of dependence of the solution of a PDE at position
$\mathbf{x}$ and at time $t$ is the set of points at a previous
time that influence the solution at position $\mathbf{x}$ and at
time $t$. }

\step{If the wind is blowing from the west at 10~miles/hour, the
following points are in the domain of dependence of Cambridge at 10am:}\step{
\begin{itemize}
\item 10 miles to the west at 9am
\item \step{15 miles to the west at ... }\opttext{8:30am}\step{
\item \opttext{20 miles to the west at 8am}}
\end{itemize}
}}

\clearpage{}

\stepwise{

Remember the analytic solution of the 1D linear advection equation:
\[
\phi(x,t)=\opttext{\phi(x-ut,0)}
\]
}Draw the domain of dependence of $x=12$m, $t=8$s on the graph,
assuming a wind-speed of $1.5$m/s:

\ifthenelse{\boolean{@studentversion}\and\not\boolean{@onlineversion}}
{
    \resizebox{0.8\textwidth}{!}{\input{figs/dod1.pdf_t}}
}
{
\stepwise*{\overlays{2}{\resizebox{0.8\textwidth}{!}{\input{figs/dod\thesubstep.pdf_t}}}}
}

\clearpage{}


\subsection{Courant-Friedrichs-Lewy (CFL) criterion:}

The domain of dependence of the numerical solution should include
the domain of dependence of the original PDE.
\begin{itemize}
\item The CFL criterion is necessary but not sufficient
\item For linear advection, the domain of dependence of the differential
equation at $(j\Delta x,n\Delta t)$ is the straight line of slope
$1/u$ through $(j\Delta x,n\Delta t)$ for $t\le n\Delta t$ in the
$(x,t)$ plane. \pause
\end{itemize}

\subsection{Domain of Dependence of FTBS}

\stepwise{$\text{FTBS:}~~~~~~~~\phi_{j}^{(n+1)}=\opttext{\phi_{j}^{(n)}-c(\phi_{j}^{(n)}-\phi_{j-1}^{(n)})}$}


\paragraph*{\textmd{$\phi_{j}^{(n)}$ depends on $\phi_{j}^{(n-1)}$ and $\phi_{j-1}^{(n-1)}$.
In turn, these depend on $\phi_{j}^{(n-2)}$, $\phi_{j-1}^{(n-2)}$
and $\phi_{j-2}^{(n-2)}$.\pause}}

\stepwise*{\textbf{Exercise: }Mark the dots that make up the domain
of dependence of $\phi_{j}^{(n+1)}$ for FTBS. Draw lines corresponding
to the real (physical) domain of dependence for cases when $c=-1,0,1,2$.
What can you deduce?\\
\ifthenelse{\boolean{@studentversion}\and\not\boolean{@onlineversion}}
{
    \resizebox{0.6\textwidth}{!}
    {\includegraphics[width=\linewidth]
    {figs/domainDependenceFTBS1.pdf}}
}
{
    \overlays{2}{\resizebox{0.6\textwidth}{!}
    {\includegraphics[width=\linewidth]
    {figs/domainDependenceFTBS\thesubstep.pdf}}}
}

\opttext{The numerical domain of dependence contains the physical
domain of dependence when $0\le c\le1$ so FTBS is unstable for $c>1$
and $c<0$. But we cannot say if FTBS is ever stable.}}\clearpage{}


\subsection{The Domain of Dependence of the CTCS Scheme}

\stepwise{


\paragraph*{
\[
\text{CTCS:}~~~~~~~~~~~~\phi_{j}^{(n+1)}=\opttext{\phi_{j}^{(n-1)}-c(\phi_{j+1}^{(n)}-\phi_{j-1}^{(n)})}
\]
}

Draw the domain of dependence of $\phi_{j}^{(n+1)}$ for CTCS.

\ifthenelse{\boolean{@studentversion}\and\not\boolean{@onlineversion}}
{
    \resizebox{0.6\textwidth}{!}
    {\includegraphics[width=\linewidth]
    {figs/domainDependenceCTCS1.pdf}}
}
{
    \overlays{2}{\resizebox{0.6\textwidth}{!}
    {\includegraphics[width=\linewidth]
    {figs/domainDependenceCTCS\thesubstep.pdf}}}
}

\step{For what values of $c$ will CTCS be unstable? }\opttext{$c<-1$ and $c>1$}

\step{Except at the initial time, the solution is found on two sets
of points that are not coupled. The solution can oscillate between
two unrelated solutions. This is a manifestation of the computational
mode of CTCS.}}

\clearpage{}


\section{Von-Neumann Stability Analysis for Wave Equations}

\liststepwise{
\begin{itemize}
\item The solution of a PDE in one spatial dimension can be expressed as
a sum of Fourier modes:
\begin{equation}
\phi=\sum_{k=-\infty}^{\infty}A_{k}e^{ikx}
\end{equation}
each with wavenumber $k$. 
\item \step{Consider the stability of a solution for individual wavenumbers}
\item \step{For a uniform grid, $x=j\Delta x$}
\item \step{Substitute $\phi_{j}^{(n)}=A^{n}e^{ikj\Delta x}$ into the
equation for a linear numerical scheme.}
\item \step{Rearrange to give an equation for the amplification factor,
$A(k,\Delta x,\Delta t)$, which tells us the following about the
scheme:}
\end{itemize}
}\liststepwise{

\global\long\def\arraystretch{1.1}
\begin{tabular}{lll}
$|A|<1$  & $\forall\ k,\ \Delta x$ & \opttext{stable and damping}\tabularnewline
$|A|=1$  & $\forall\ k,\ \Delta x$ & \opttext{neutrally stable}\tabularnewline
$|A|>1$  & for any $k,\ \Delta x$ & \opttext{unstable (amplifying)}\tabularnewline
\end{tabular}

Where $|A|^{2}=AA^{*}$ ($A$ multiplied by its complex conjugate).
\begin{itemize}
\item \step{For linear advection $A$ should be complex since the solution
changes location every time-step}
\end{itemize}
}\clearpage{}


\subsection{Von-Neumann Stability Analysis of FTBS}

\stepwise{

FTBS for linear advection is: 
\begin{equation}
\phi_{j}^{(n+1)}=\phi_{j}^{(n)}-c\bigl(\phi_{j}^{(n)}-\phi_{j-1}^{(n)}\bigr)
\end{equation}
Substitute in $\phi_{j}^{(n)}=A^{n}e^{ikj\Delta x}$ :

\begin{equation}
\opttext{A^{n+1}e^{ikj\Delta x}=A^{n}e^{ikj\Delta x}-cA^{n}(e^{ikj\Delta x}-e^{ik(j-1)\Delta x})}
\end{equation}
Cancel powers of $A^{n}e^{ikj\Delta x}$ and rearrange to find $A$
in terms of $c$ and $k\Delta x$: 
\begin{equation}
\opttext{A=1-c(1-e^{-ik\Delta x})}
\end{equation}
We need to find the magnitude of $A$ so we need to write it down
in real and imaginary form. So substitute $e^{-ik\Delta x}=\cos k\Delta x-i\sin k\Delta x$:
\begin{equation}
\opttext{A=1-c(1-\cos k\Delta x)-ic\sin k\Delta x}
\end{equation}
\step{and calculate $|A|^{2}=AA^{*}$ ($A$ multiplied by its complex
conjugate):
\begin{align*}
|A|^{2} & =1-2c(1-\cos k\Delta x)+c^{2}(1-2\cos k\Delta x+\cos^{2}k\Delta x)+c^{2}\sin^{2}k\Delta x\\
\implies|A|^{2} & =1-2c(1-c)(1-\cos k\Delta x)
\end{align*}
}\step{We need to find for what value of $\Delta t$ or $c$ is $|A|\le1$
in order to find when FTBS is stable: 
\begin{align*}
|A|\le1 & \iff|A|^{2}-1\le0\\
 & \iff-2c(1-c)(1-\cos k\Delta x)\le0\\
 & \iff c(1-c)(1-\cos k\Delta x)\ge0\\
\end{align*}
}\step{So we know that FTBS is stable only when $c(1-c)(1-\cos k\Delta x)\ge0$
and that this conditions must hold regardless of the value of $k\Delta x$.
Let us plot the curve for some values of $k\Delta x$:}}

\clearpage{}\liststepwise*{\textbf{Exercise: }Sketch $c(1-c)(1-\cos k\Delta x)$
for $k\Delta x=\pm\pi$, $k\Delta x=\pm\pi/2$, $k\Delta x=\pm\pi/4$,
$k\Delta x=0$

\ifthenelse{\boolean{@onlineversion}}%
{
    \overlays{5}{\resizebox{0.6\textwidth}{!}%
    {\includegraphics[width=\linewidth]
    {python/FTBSstability/FTBSstability\thesubstep.pdf}}}
}
{
    \ifthenelse{\boolean{@studentversion}}%
    {%
\includegraphics[width=0.6\linewidth]{python/FTBSstability/FTBSstability1.pdf}%
    }%
    {
\includegraphics[width=0.6\linewidth]{python/FTBSstability/FTBSstability5.pdf}%
    }
}

\step{$k\Delta x$ is set by the initial conditions. So stability
should not depend on $k\Delta x$.

So FTBS is stable when...}\opttext{ $0\le c\le1$.}

\step{We have proved that FTBS is unstable if $u<0$ or if $\frac{u\Delta t}{\Delta x}>1$.
We will now define:\\
 %
\begin{tabular}{lll}
\textbf{Upwind scheme } & FTBS  & when $u\ge0$\tabularnewline
 & FTFS  & when $u<0$ \tabularnewline
\end{tabular}

\textbf{\small{}The upwind scheme is first order accurate in space
and time, conditionally stable and damping.} }\step{

\textcolor{blue}{Question:} For what values of $\Delta t$ is the
upwind scheme stable? }\opttext{$\Delta t\le\Delta x/|u|$}

\step{

\textcolor{blue}{Question:} How does this compare with the conclusion
from the domain of dependence?}}

\clearpage{}


\subsection{Von Neumann Stability Analysis of CTCS}

\liststepwise{

The amplification factor, $A$, is found by substituting $\phi_{j}^{(n)}=\opttext{A^{n}e^{ikj\Delta x}}$\\
into the equation for CTCS: 
\[
\phi_{j}^{(n+1)}=\opttext{\phi_{j}^{(n-1)}-c\bigl(\phi_{j+1}^{(n)}-\phi_{j-1}^{(n)}\bigr).}
\]
This gives: 
\begin{align*}
\  & \opttext{A^{n+1}e^{ikj\Delta x}-A^{n-1}e^{ikj\Delta x}+c(A^{n}e^{ik(j+1)\Delta x}-A^{n}e^{ik(j-1)\Delta x})=0} & \  & \ \\
\  & \opttext{A^{2}+(2ic\sin k\Delta x)A-1=0} & \  & \ \\
\implies & A=\opttext{-ic\sin k\Delta x\pm\sqrt{1-c^{2}\sin^{2}k\Delta x}} & \  & \ 
\end{align*}


\step{There are two cases to consider:}

{\setlength{\tabcolsep}{2pt} \global\long\def\arraystretch{1.5}
 %
\begin{tabular}{ll}
\step{(i) $|c|\le1$  & $\implies c^{2}\sin^{2}k\Delta x\le1$ }\tabularnewline
 & $\step{\implies|A|^{2}=}\opttext{1-c^{2}\sin^{2}k\Delta x+c^{2}\sin^{2}k\Delta x=1}$\tabularnewline
 & $\step{\implies}$ \opttext{The solution is stable and not damping}\tabularnewline
\step{(ii) $|c|>1$  & $\implies c^{2}\sin^{2}k\Delta x>1$ for some $k\Delta x$}\tabularnewline
 & $\step{\implies|A|^{2}=}\opttext{(c\sin k\Delta x\pm\sqrt{c^{2}\sin^{2}k\Delta x-1})^{2}}$\tabularnewline
 & $\step{\implies}$\opttext{ At least one of the roots has $|A|>1$.
The solution is unstable}\tabularnewline
\end{tabular}}

\step{So CTCS is conditionally stable: it is stable for ... }\opttext{$|c|\le1$}

\step{Regardless of $c$, there are always two possible values of
$A$. This means that there will always be two possible solutions;
a realistic solution (the physical mode) and the un-realistic, oscilating
solution; the spurious computational mode}

}\clearpage{}


\section{Exercises: Analysis of Advection Schemes}
\begin{enumerate}
\item Write down the following schemes and derive expressions for their
amplification factors:

\begin{enumerate}
\item FTCS 
\item BTCS 
\item CTBS 
\end{enumerate}
\item Which of the above schemes suffer from a spurious computational mode?
\item Sketch of the domain of dependence for these schemes 
\item For FTCS and BTCS

\begin{enumerate}
\item Find for what Courant numbers the schemes are stable.
\item Are the schemes damping, amplifying or neutral?
\item Write down an expressions for the phase speeds of the numerical waves
(not covered)
\end{enumerate}
\item Determine if mass is conserved by CTCS (not covered)
\end{enumerate}
\optpage{


\section*{Answers}
\begin{enumerate}
\item Expressions for the amplification factors for the following advection
schemes: 

\begin{enumerate}
\item FTCS: $\phi_{j}^{(n+1)}=\phi_{j}^{(n)}-\frac{c}{2}\left(\phi_{j+1}^{(n)}-\phi_{j-1}^{(n)}\right)$\\
Substitute in $\phi_{j}^{(n)}=A^{n}e^{ikj\Delta x}$ and cancel powers
of $A^{n}e^{ikj\Delta x}$:
\begin{align*}
A= & 1-\frac{c}{2}(e^{ik\Delta x}-e^{-ik\Delta x})\\
= & 1-ic\sin k\Delta x
\end{align*}

\item BTCS: $\phi_{j}^{(n+1)}=\phi_{j}^{(n)}-\frac{c}{2}\left(\phi_{j+1}^{(n+1)}-\phi_{j-1}^{(n+1)}\right)$\\
Substitute in $\phi_{j}^{(n)}=A^{n}e^{ikj\Delta x}$ and cancel powers
of $A^{n}e^{ikj\Delta x}$:
\begin{align*}
A= & 1-\frac{c}{2}A(e^{ik\Delta x}-e^{-ik\Delta x})\\
\Rightarrow A= & \frac{1}{1+ic\sin k\Delta x}\\
= & \frac{1-ic\sin k\Delta x}{1+c^{2}\sin^{2}k\Delta x}
\end{align*}

\item CTBS: $\phi_{j}^{(n+1)}=\phi_{j}^{(n-1)}-2c\left(\phi_{j}^{(n)}-\phi_{j-1}^{(n)}\right)$\\
Substitute in $\phi_{j}^{(n)}=A^{n}e^{ikj\Delta x}$ and cancel powers
of $A^{n}e^{ikj\Delta x}$:
\begin{align*}
A= & \frac{1}{A}-2c(1-e^{-ik\Delta x})\\
\Rightarrow & A^{2}+2c(1-\cos k\Delta x+i\sin k\Delta x)A-1=0\\
\Rightarrow A= & -2c(1-\cos k\Delta x+i\sin k\Delta x)\pm\sqrt{c^{2}(1-\cos k\Delta x+i\sin k\Delta x)^{2}+1}
\end{align*}

\end{enumerate}
\item CTBS will suffer from a spurious computational mode because there
are two possible solutions for $A$ whereas there is only one possible
solution for the first order linear advection PDE.
\item Domain of dependence for:\\
\begin{tabular}{ccc}
(a) FTCS & (b) BTCS & (c) CTBS\tabularnewline
\includegraphics[width=0.3\linewidth]{figs/domainDependenceFTCS} & \includegraphics[width=0.3\linewidth]{figs/domainDependenceBTCS} & \includegraphics[width=0.3\linewidth]{figs/domainDependenceCTBS}\tabularnewline
\end{tabular}
\item For FTCS:

\begin{enumerate}
\item Find for what Courant numbers the scheme is stable: 
\begin{align*}
A= & 1-ic\sin k\Delta x\\
\Rightarrow|A|^{2}= & 1+c^{2}\sin^{2}k\Delta x
\end{align*}
which is $>1$ for all $|c|>0$ and so FTCS is unconditioanlly unstable
\item FTCS is amplifying since $A>1$ for all $|c|>0$
\item The phase speeds of the numerical waves is
\[
u_{c}=\frac{\tan^{-1}\left(c\sin k\Delta x\right)}{k\Delta t}
\]

\end{enumerate}

For BTCS:
\begin{enumerate}
\item Find for what Courant numbers the scheme is stable: 
\begin{align*}
A= & \frac{1}{1+ic\sin k\Delta x}\\
\Rightarrow|A|^{2}= & \frac{1}{1+c^{2}\sin^{2}k\Delta x}\le1\forall|c|>0
\end{align*}
so BTCS is unconditioanlly stable
\item BTCS is damping since $A<1$ for all $|c|>0$
\item The phase speeds of the numerical waves is
\[
u_{c}=\frac{\tan^{-1}\left(c\sin k\Delta x\right)}{k\Delta t}
\]

\end{enumerate}
\item Determine if mass is conserved by CTCS:
\[
\text{CTCS: }\ \ \phi_{j}^{(n+1)}=\phi_{j}^{(n-1)}-c\bigl(\phi_{j+1}^{(n)}-\phi_{j-1}^{(n)}\bigr).
\]
From this we can calculate $M^{(n+1)}$ as a function of $M^{(n-1)}$:
\begin{align*}
M^{(n+1)} & =\sum_{j=1}^{n_{x}}\phi_{j}^{(n+1)}=\sum_{j=1}^{n_{x}}\phi_{j}^{(n-1)}-c\bigl(\phi_{j+1}^{(n)}-\phi_{j-1}^{(n)}\bigr)=M^{(n-1)}-c\left(\sum_{j=1}^{n_{x}}\phi_{j+1}^{(n)}-\sum_{j=1}^{n_{x}}\phi_{j-1}^{(n)}\right)\\
 & =M^{(n-1)}-c\left(\sum_{j=2}^{n_{x}+1}\phi_{j}^{(n)}-\sum_{j=0}^{n_{x}-1}\phi_{j}^{(n)}\right)=M^{(n-1)}-c\left(\phi_{n_{x}}^{(n)}+\phi_{n_{x}+1}^{(n)}-\phi_{0}^{(n)}-\phi_{1}^{(n)}\right)=M^{(n-1)}
\end{align*}
due to the periodic boundaries since $\phi_{n_{x}}^{(n)}=\phi_{0}^{(n)}$
and $\phi_{n_{x}+1}^{(n)}=\phi_{1}^{(n)}$. Therefore $M^{(n+1)}=M^{(n-1)}$
which means that mass is conserved.
\end{enumerate}
}
