\clearpage{}


\section{Practical: Shallow-Water Equations (SWE)}

In this practical you will develop existing code for solving the SWE
in 2D or the Arakawa A- and C-grids. You will add a mountain and non-linear
terms in order to see the influence on the fluid dynamics and on numerical
artifacts of the solution. Given time or collaboration, you may be
able to compare different numerical methods.

Code to solve the linearised shallow-water equations:
\begin{eqnarray}
\frac{\partial\mathbf{u}}{\partial t} & +2\bm{\Omega}\times\mathbf{u} & =-g\nabla\left(h+h_{0}\right)\\
\frac{\partial h}{\partial t} & +\nabla\cdot\left(\left(H-h_{0}\right)\mathbf{u}\right) & =0
\end{eqnarray}
using the Arakawa A and C-grids are in the files \url{practicals/SWE/*.py}
. The (horizontal) velocity is $\mathbf{u}=(u,v,0)^{T}$, $\bm{\Omega}=(0,0,2f)^{T}$
is the rotation of the planar geometry, where $f=\beta y$ is the
Coriolis parameter, $g$ is gravity, $h$ is the depth anomaly in
comparison to the mean depth, $H$ and $h_{0}$ is the height of the
bottom topography. The boundary conditions in the $x$ direction are
periodic and in the $y$ direction the boundary conditions are $v=0$
and $\partial h/\partial y$=0. The code is set up to solve a test
case with a geostrophically balanced jet described by: 
\begin{eqnarray*}
u(\hat{y}) & = & \begin{cases}
u_{m}\left(1-3\hat{y}^{2}+3\hat{y}^{4}-\hat{y}^{6}\right) & \text{for }-1\le\hat{y}\le1\\
0 & \text{otherwise}
\end{cases}\\
v & = & 0\\
h\left(\hat{y}\right) & = & \begin{cases}
16\ w\beta u_{m}y_{c}/(35g) & \textnormal{for }\hat{y}\le-1\\
-w\beta u_{m}/g\begin{array}[t]{l}
\bigl\{ y_{c}\left(\hat{y}-\hat{y}^{3}+3\hat{y}^{5}/5-\hat{y}^{7}/7\right)\\
+w\left(-1/8+\hat{y}^{2}/2-3\hat{y}^{4}/4+\hat{y}^{6}/2-\hat{y}^{8}/8\right)\bigr\}
\end{array} & \textnormal{for }-1\le\hat{y}\le1\\
-16\ w\beta u_{m}y_{c}/(35g) & \textnormal{for }\hat{y}\ge1
\end{cases}
\end{eqnarray*}
where $\hat{y}=(y-y_{c})/w$, $y_{c}$ is the $y$ position of the
jet centre, $w$ is the jet half width, $u_{m}$ is the maximum jet
velocity and $h_{0}$ is the minimum height. The default parameters
are given in table \ref{tab:SWEparameters} and also in the code.

\noindent \begin{center}
\begin{table}
\renewcommand{\arraystretch}{1.2}

\noindent \begin{centering}
\begin{tabular}{|l|c|c|c|}
\hline 
\textbf{Parameter} &  & value & units\tabularnewline
\hline 
\hline 
\textbf{Fluid parameters:} &  &  & \tabularnewline
\hline 
Variation of Coriolis with latitute & $\beta$ & $2\times10^{-11}$ & $\text{m}^{-1}\text{s}^{-1}$\tabularnewline
\hline 
Gravity & $g$ & 10 & $\text{m}\text{s}^{-2}$\tabularnewline
\hline 
Mean fluid depth & H & 3000 & m\tabularnewline
\hline 
\textbf{Geometry and grid parameters:} &  &  & \tabularnewline
\hline 
Domain size in $x$ direction & $x_{max}$ & $1.2\times10^{7}$ & m\tabularnewline
\hline 
Domain size in $y$ direction & $y_{max}$ & $1.2\times10^{7}$ & m\tabularnewline
\hline 
Grid spacing in $x$ direction & $\Delta x$ & $x_{max}/39$ & m\tabularnewline
\hline 
Grid spacing in $y$ direction & $\Delta y$ & $y_{max}/39$ & m\tabularnewline
\hline 
\textbf{Jet parameters:} &  &  & \tabularnewline
\hline 
Jet centre & $y_{c}$ & $6\times10^{6}$ & m\tabularnewline
\hline 
Jet width & $w$ & $3\times10^{6}$ & m\tabularnewline
\hline 
Maximum jet velocity & $u_{m}$ & 20 & $\text{m}\text{s}^{-1}$\tabularnewline
\hline 
\end{tabular}
\par\end{centering}

\caption{Default parameter values for the geostrophically balanced jet\label{tab:SWEparameters}}
\end{table}

\par\end{center}

Some students should make developments to the A-grid code and others
to the C-grid code. Discuss with your colleagues to decide you does
which. If you have previously developed a C-grid code, then develop
an A-grid code now. Or for an alternative challenge, write B, D or
E grid codes. 
\begin{enumerate}
\item Run the A- and C-grid versions of the shallow-water code. The errors
for this case can easily be calculated by comparing the initial and
final conditions since the initial conditions are the analytic solution.

\begin{enumerate}
\item Calculate and plot the errors after running for 1 day.
\item Why are the C-grid results are more accurate than the A-grid results? 
\item Can more accurate results be obtained for lower computational cost
by independently varying $n_{x}$, $n_{y}$, and $\Delta t$?
\end{enumerate}
\item In the code in \url{practicals/SWE/*.py}, the bottom topography has
not been fully implemented. Add a mountain defined by: 
\[
h_{0}=\begin{cases}
h_{0\max}(1-\sqrt{(x-x_{c})^{2}+(y-y_{c})^{2}}/r_{m}) & \text{ for }\sqrt{(x-x_{c})^{2}+(y-y_{c})^{2}}<r_{m}\\
0 & \text{ otherwise}
\end{cases}
\]
where $(x_{c},y_{c})$ is the mountain centre which is in the middle
of the domain, $r_{m}=1.5\times10^{6}\text{m}$ is the mountain radius
and $h_{0\max}=500\text{m}$ is the mountain peak height. You will
need to add the definition of the mountain to function \url{createMountain}
in file \url{practicals/SWE/initialAndPlot.py} and include the influence
of the orography in the correct terms of the SWE in functions \url{SWECgrid}
or \url{SWEAgrid} in file \url{practicals/SWE/ACgrids.py}. 

\begin{enumerate}
\item Plot the results for both schemes after 6 days. 
\item Can you explain the influence of the mountain on the flow?
\item Can you explain the problems with the A-grid results?
\item Is it beneficial to have isotropic resolution ($\Delta x=\Delta y$)
or anisotropic when a mountain is present?
\end{enumerate}
\item For what values of the advective and gravity wave Courant number is
the code stable? Modify the parameters so that the advective Courant
number is smaller than the gravity wave Courant number. Which Courant
number(s) control stability?
\item Design and write down a discretisation (on the A- or C-grid) for the
non-linear SWE in the form:
\begin{eqnarray}
\frac{\partial\mathbf{u}}{\partial t} & +\mathbf{u}\cdot\nabla\mathbf{u}+2\bm{\Omega}\times\mathbf{u} & =-g\nabla\left(h+h_{0}\right)\\
\frac{\partial h}{\partial t} & +\nabla\cdot\left(h\mathbf{u}\right) & =0
\end{eqnarray}
where $h$ is now the total fluid height. Write a new function to
solve these equations. Re-run the test cases with and without the
mountain using the non-linear equation solver. (Use the original parameter
values from table \ref{tab:SWEparameters}.)


\textbf{Additional questions to consider}

\item Write a functions to calculate the vorticity and divergence of the
solution and the change from the initial conditions. What kind of
waves are revealed by these diagnostics. Do these waves behave differently
when solving the linearised or non-linear equations.
\item For what values of the advective and gravity wave Courant number is
the non-linear code stable? How and why has this changed from the
linear equation solver.
\item What additional problems are encountered when solving the non-linear
equations. How could this be dealt with?\end{enumerate}

