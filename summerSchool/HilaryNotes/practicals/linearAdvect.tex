%
\cleardoublepage{}


\section{Practical: Linear Advection}

You will be coding up various numerical methods in order to understand
how they behave, which schemes are more accurate and what problems
and limitations can occur. You will be writing code to solve the one-dimensional
linear advection equation using either a uniform or non-uniform grid,
depending on your confidence and experience. Some code has been written
already and you can use this to get you started. Some of the code
is available in printed form only. The rest is available in directory
\url{practicals/linearAdvectCode}. (Evidence shows that copying out
code is a good way to learn programming). The code is organised as
follows:

\begin{tabular}{ll}
\url{linearAdvectUni.py} & The outer code for linear advection using a uniform grid\tabularnewline
 & to define variables, call advection functions and plot results\tabularnewline
\url{linearAdvectNonU.py} & Similar outer code but using a non-uniform grid\tabularnewline
\url{spaceTime.py} & Code for defining data structures \texttt{Space} and \texttt{Time}\tabularnewline
\url{mySchemes.py} & File containing various advection schemes for you to write.\tabularnewline
 & Rename this file \url{yourName.py} so that all files are unique\tabularnewline
\url{initialConditions.py} & For defining various initial conditions, digital copy available\tabularnewline
\url{diagnostics.py} & For calculating error diagnostics, digital copy available\tabularnewline
\end{tabular}

\textbf{You do not need to type out both the uniform and non-uniform
grid code}. 
\begin{itemize}
\item Choose a scheme to implement, FTBS, CTCS, BTBS etc or a scheme you
have found online or one you have derived yourself. I have already
implemented FTCS on a uniform and non-uniform grid to help you get
started.
\item Choose if you are going to implement your scheme on a uniform or non-uniform
grid. 
\item Discuss your decisions with your colleagues so that a range of different
schemes are implemented. This will enable you to share code and study
the behaviour of different schemes. 
\item Implement your chosen scheme in \url{yourName.py} and run to compare
with FTCS. 
\item Share code with your neighbours to study the behaviour of different
schemes. 
\end{itemize}
The codes should use periodic boundary conditions so that when material
is advected out of the right hand side (at $x=1$), it appears at
the left hand boundary (at $x=0$) and vice versa. This can be done
by setting $\phi_{N}=\phi_{0}$ , $\phi_{N+1}=\phi_{1}$, $\phi_{-1}=\phi_{N-1}$.
However, to simplify the code, this is done in python using modulo
arithmetic to calculate array indices. See code for my implementation
of FTCS in \url{mySchemes.py}.

You will have the opportunity to present the results of your work
to the whole group. So in collaboration with your colleagues, make
some figures and some slides which answer the following points concerning
the behaviour of the schemes:
\begin{enumerate}
\item Which schemes have the lowest errors?
\item How do the errors vary with $\Delta x$?
\item How do the errors change if $\Delta x$ is non-uniform?

\begin{enumerate}
\item Does reducing $\Delta x$ in part of the domain improve accuracy?
\end{enumerate}
\item Which schemes(s) conserve the total quantity of $\phi$?
\item Which schemes(s) conserve the standard deviation of $\phi$?
\item Are there any stability limits associated with any of the schemes?
(Are the schemes stable for $c<0$, $c<1$ or $c>1$?) (If a scheme
is stable then it should be able to run for an arbitrarily long time
without diverging.)
\item Do any of the schemes generate unbounded results (values $<0$ or
$>1$)?
\end{enumerate}

\subsection*{\clearpage{}Code for \protect\url{spaceTime.py}}

\lstinputlisting{practicals/linearAdvectCode/spaceTime.py}\clearpage{}


\subsection*{Code for \protect\url{linearAdvectUni.py}}

\lstinputlisting{practicals/linearAdvectCode/linearAdvectUni.py}


\subsection*{\clearpage{}Code for \protect\url{mySchemes.py}}

\lstinputlisting{practicals/linearAdvectCode/mySchemes.py}


\subsection*{\clearpage{}Code for \protect\url{linearAdvectNonU.py}}

\lstinputlisting{practicals/linearAdvectCode/linearAdvectNonU.py}%

