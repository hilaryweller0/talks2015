
\chapter{Alternative Advection Schemes}

\liststepwise{

Conclusions from practicals and lectures on advection so far:
\begin{itemize}
\item \opttext{First-order in space schemes are very diffusive}\opttext{
\item Second-order in space, linear schemes lead to grid-scale oscillations}\opttext{
\item Explicit Eulerian schemes have time-step restrictions }\opttext{
\item Changes in resolution can make errors larger}
\end{itemize}
\vfill{}


\step{


\subsection*{Some possible solutions to these problems}
\begin{enumerate}
\item Semi-Lagrangian advection
\item Artificial diffusion to remove spurious oscillations 
\item Total variation diminishing (TVD) schemes
\end{enumerate}
}}

\clearpage{}


\section{Semi-Lagrangian Advection}

\liststepwise{

\textbf{CFL criterion}: The domain of dependence of the numerical
solution must contain the domain of dependence of the original PDE.
\begin{itemize}
\item So to allow long time-steps, construct the numerical domain of dependence
to contain the physical domain of dependence.\step{
\item Recall the linear advection equation: ${\textstyle \frac{\partial\phi}{\partial t}+u\frac{\partial\phi}{\partial x}=0}$
\item and its analytic solution: $\phi(x,t)=\phi(x-ut,0)$}\step{


\hfill{}\raisebox{-0.15\linewidth}[0pt][0pt]{\includegraphics[width=0.4\linewidth]{/home/hilary/latex/meetings/2013/SS/HilaryNotes/figs/SL}}

\item Semi-Lagrangian advection is defined from this:\\
$\phi(x_{j},t_{n+1})=\phi(x_{j}-u\Delta t,t_{n})=\phi(x_{jd},t_{n})$\\
$x_{jd}=x_{j}-u\Delta t$ is the departure point of point $x_{j}$.}\step{
\item So interpolate $\phi$ from known points onto $x_{jd}$.}\step{
\item First find $k$ such that $x_{k}\le x_{jd}\le x_{k+1}$: (Use $\text{int}(x)$
meaning the integer part of $x$) \\
$k=\opttext{\text{int}((x_{j}-u\Delta t)/\Delta x)=\text{int}(j-c)}$}\step{
\item Interpolate from $x_{k-1}$, $x_{k}$, $x_{k+1}$, $x_{k+2}$, ...
onto $x_{jd}$ \\
(for example using cubic-Lagrange interpolation)}\step{
\item The time step is no-longer restricted by the Courant number but for
stability, a sufficiently small time step must be used so that trajectories
do not cross. }
\end{itemize}
\step{\textbf{Exercise:} Find $\beta=\frac{x_{jd}-x_{k}}{\Delta x}$
as a function of only $j$, $k$ and $c$ given $x_{j}=j\Delta x$
and $c=u\Delta t/\Delta x$. }\opttext{$\beta=j-k-c$}\step{

\textbf{Problem}: The advected quantity is not conserved}}

\clearpage{}


\section{Artificial Diffusion to Remove Spurious Oscillations}

\liststepwise{

Numerical schemes are designed not to produce spurious oscillations.
However, once a forecasting model is put together, often spurious
oscillations are still generated. These can be removed by adding an
artificial diffusion term to the equations. \step{For example, the
linear advection equation with diffusion: 
\begin{equation}
\frac{\partial\phi}{\partial t}+\vec{u}\dprod\nabla\phi-K\nabla^{2}\phi=0
\end{equation}
or in one dimension:} 
\begin{equation}
\step{\frac{\partial\phi}{\partial t}+}\opttext{u\frac{\partial\phi}{\partial x}-K\frac{\partial^{2}\phi}{\partial x^{2}}=0}
\end{equation}

\begin{itemize}
\item \step{This is dampens spurious oscillations and real features }\step{
\item It is only (but frequently) used as a last resort.}\step{
\item More scale-selective filtering can be achieved using $\nabla^{4}$
rather than $\nabla^{2}$:
\begin{equation}
\frac{\partial\phi}{\partial t}+\vec{u}\dprod\nabla\phi+K\nabla^{4}\phi=0
\end{equation}
}
\end{itemize}
\step{\textbf{Information for Practical: }The following advection-diffusion
scheme is stable for $c^{2}+4d\le1$ where $d=K\Delta t/\Delta x^{2}$:
\begin{equation}
\phi_{j}^{(n+1)}-\phi_{j}^{(n-1)}+c(\phi_{j+1}^{(n)}-\phi_{j-1}^{(n)})-2d(\phi_{j+1}^{(n-1)}-2\phi_{j}^{(n-1)}+\phi_{j-1}^{(n-1)})=0.
\end{equation}
}}\clearpage{}


\section{Total Variation Diminishing (TVD) schemes}

\liststepwise*{
\begin{itemize}
\item \step{Linear, second-order advection schemes produce unbounded, unrealistic,
grid-scale oscillations. These can be measured by the total variation:
\[
TV={\textstyle \sum_{j}|\phi_{j+1}-\phi_{j}|}
\]
}\step{\\
\textbf{Exercise:} Calculate the total variation of these functions:\renewcommand\arraystretch{-0.5}\\
\begin{tabular}{lll}
\includegraphics[width=0.3\textwidth]{python/TV/TV1} & \includegraphics[width=0.3\textwidth]{python/TV/TV2} & \includegraphics[width=0.3\textwidth]{python/TV/TV3}\tabularnewline
\includegraphics[width=0.3\textwidth]{python/TV/TV4} & \includegraphics[width=0.3\textwidth]{python/TV/TV5} & \includegraphics[width=0.3\textwidth]{python/TV/TV6}\tabularnewline
\end{tabular}\\
}\opttext{(a) 8, (b) 8, (c) 12, (d) 8, (e) 12, (f) 6}\step{
\item A total variation diminishing (TVD) scheme has $TV^{(n+1)}\le TV^{(n)}$.
}
\end{itemize}
\step{\textbf{Question:} Why is total variation used rather than
boundedness to measure the generation of spurious oscillations? ...\opttext{Because spurious oscillations can be generated within the bounds of the original function. Eg see function (e) above.}

}}

\clearpage{}

\liststepwise*{
\begin{itemize}
\item \step{First-order upwind is the only \textit{linear} TVD scheme (Godunov's
theorem). Other TVD schemes are non-linear... }\step{
\item To define a TVD scheme, start with the discretised linear advection
equation: 
\begin{equation}
\frac{\phi_{j}^{(n+1)}-\phi_{j}^{(n)}}{\Delta t}=-u\frac{\phi_{j+\half}-\phi_{j-\half}}{\Delta x}.
\end{equation}
}\step{
\item Each $\phi_{j\pm\half}$, is calculated as a weighted average of a
high order flux ($\phi_{H}$) and a low order flux ($\phi_{L}$):
\[
\phi_{j+\half}=\Psi_{j+\half}~\phi_{H}+(1-\Psi_{j+\half})~\phi_{L}
\]
where $\Psi$ is a limiter function.}\step{
\item Use as much of $\phi_{H}$ as possible without introducing oscillations
\item So $\Psi$ should be close to one where the solution is smooth so
that the solution is close to second-order accurate and $\Psi$ close
to zero where the solution changes rapidly so as to use the upwind
flux which guarantees boundedness. }\step{
\item The scheme is now non-linear since $\Psi$ depends on $\phi$. 
\item We can use:


\begin{tabular}{cll}
-- & Centred in space as the high-order flux: & $\phi_{H}=\half(\phi_{j}+\phi_{j+1})$\tabularnewline
-- & First-order upwind as the low-order flux: & $\phi_{H}=\begin{cases}
\phi_{j} & \text{if }u\ge0\\
\phi_{j1} & \text{{otherwise}}
\end{cases}$\tabularnewline
\end{tabular}


}\step{

\item So what should $\Psi$ be?}
\end{itemize}
}

\clearpage{}


\subsection{Limiter functions}

\liststepwise*{\step{

The limiter function, $\Psi$, is based on the ratio of the upwind
gradient to the local gradient: 
\[
r_{j+\half}=\frac{\phi_{j}-\phi_{j-1}}{\phi_{j+1}-\phi_{j}}
\]
for $u>0$.}\step{


\subsubsection*{An Example: the Van Leer Limiter}

\[
\Psi(r)=\left(r+|r|\right)/\left(1+|r|\right)
\]
}\step{A lot of algebra will show you that this scheme is TVD.}\step{

There are many other limiter functions that give TVD schemes.

See, eg \url{https://en.wikipedia.org/wiki/Flux_limiter}

}}
