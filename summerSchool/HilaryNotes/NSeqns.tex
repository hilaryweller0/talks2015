
\chapter{The Navier Stokes Equations}

The Navier-Stokes Equations for a compressible, rotating atmosphere

{\global\long\def\arraystretch{0.8}


\begin{tabular}{ll}
\multicolumn{2}{l}{\textbf{The Lagrangian derivative:} $\frac{D\Psi}{Dt}=\frac{\partial\Psi}{\partial t}+\mathbf{u}\cdot\nabla\Psi$}\tabularnewline
 & \tabularnewline
Momentum  & $\frac{D\mathbf{u}}{Dt}=\text{-}2\bm{\Omega}\times\mathbf{u}-\frac{\nabla p}{\rho}+\mathbf{g}+\mu_{u}\left(\nabla^{2}\mathbf{u}+\frac{1}{3}\nabla(\nabla\cdot\mathbf{u})\right)$ \tabularnewline
 & \tabularnewline
Continuity  & $\frac{D\rho}{Dt}+\rho\nabla\cdot\mathbf{u}=0$\tabularnewline
 & \tabularnewline
Energy  & $\frac{D\theta}{Dt}=Q+\mu_{\theta}\nabla^{2}\theta$\tabularnewline
 & \tabularnewline
\multicolumn{2}{l}{An equation of state, eg perfect gas law, $p=\rho RT$}\tabularnewline
\end{tabular}}{\global\long\def\arraystretch{1}


\noindent \begin{flushleft}
\begin{tabular}{llll}
$\mathbf{u}$  & Wind vector  & $\vec{g}$  & Gravity vector (downwards)\tabularnewline
$t$  & Time  & $\theta$  & Potential temperature, $T\left(p_{0}/p\right)^{\kappa}$\tabularnewline
$\bm{\Omega}$  & Rotation rate of planet & $\kappa$ & heat capacity ratio $\approx1.4$\tabularnewline
$\rho$  & Density of air & $Q$ & Source of heat\tabularnewline
$p$  & Atmospheric pressure & $\mu_{u}$, $\mu_{\theta}$  & Diffusion coefficients\tabularnewline
\end{tabular}
\par\end{flushleft}

}\pause 
\begin{itemize}
\item What does $\mathbf{u}$ mean?\pause 
\item What does $\nabla\cdot\mathbf{u}$ mean?\pause 
\item What does $\nabla p$ mean?\pause 
\item What does $\mathbf{u}\cdot\nabla\Psi$ mean?
\end{itemize}

\section{The Potential Temperature Equation}

\begin{center}
\begin{tabular}{ccccccccc}
$\frac{D\theta}{Dt}$ & $=$ & $\frac{\partial\theta}{\partial t}$ & $+$ & $\mathbf{u}\cdot\nabla\theta$ & $=$ & $Q$ & $+$ & $\mu_{\theta}\nabla^{2}\theta$\tabularnewline
Lagrangian &  & Rate of change &  & Advection &  & Heat &  & Diffusion\tabularnewline
derivative &  & at fixed point &  & of $\theta$ &  & source &  & of $\theta$\tabularnewline
\end{tabular}
\par\end{center}

\pause%
\begin{minipage}[c]{0.35\columnwidth}%
\begin{flushleft}
$\theta$ will be created and destroyed by the heat source, $Q$ ,
it will be moved around by the wind field, $\mathbf{u}$, and $\theta$
will be diffused by a diffusion coefficient, $\mu_{\theta}$
\par\end{flushleft}%
\end{minipage}%
\begin{minipage}[c]{0.6\columnwidth}%
\mediaMovie[autostart,loop]
{\includegraphics[width=\linewidth]{/home/hilary/OpenFOAM/hilary-2.3.0/run/teachingAnims/advection/movie/T1.jpg}}
{animations/teachingAnims_advection_movie_T.mp4}%
\end{minipage}


\section{Advection of Pollution\label{sec:Advection-of-Pollution}}


\subsection{Pure Linear Advection}

Advection of concentration $\phi$ without diffusion or sources or
sinks:
\begin{equation}
\frac{D\phi}{Dt}=\frac{\partial\phi}{\partial t}+\mathbf{u}\cdot\nabla\phi=0\label{eqn:3dAdvect}
\end{equation}
Changes of $\phi$ are produced by the component of the wind in the
same direction as gradients of $\phi$. In order to understand why
the $\mathbf{u}\cdot\nabla\phi$ term leads to changes in $\phi$,
consider a region of polluted atmosphere where the pollutant has the
concentration contours shown below:

\noindent %
\begin{minipage}[c]{0.5\linewidth}%
 \vspace{1cm}
 \includegraphics[width=1\linewidth]{/home/hilary/latex/teaching/MTMW12/2014/notes/figs/contours} %
\end{minipage}\hfill{}%
\begin{minipage}[c]{0.43\linewidth}%
\textbf{Exercise}: Draw on the figure the directions of the gradients
of $\phi$ and thus mark with a $+$, $-$ or $0$ locations where
$\mathbf{u}\cdot\nabla\phi$ is positive, negative and zero. Thus
deduce where $\phi$ is increasing, decreasing or staying the same
based on equation \ref{eqn:3dAdvect}. Hence overlay contours of $\phi$
at a later time. %
\end{minipage}

\clearpage{}


\subsection{Advection/Diffusion with Sources and Sinks\label{sub:AdvDiff}}

\begin{center}
\begin{tabular}{ccccccccc}
$\frac{D\Psi}{Dt}$ & $=$ & $\frac{\partial\Psi}{\partial t}$ & $+$ & $\mathbf{u}\cdot\nabla\Psi$ & $=$ & $S$ & $+$ & $\mu_{\Psi}\nabla^{2}\Psi$\tabularnewline
Lagrangian &  & Rate of change &  & Advection &  & Sources &  & Diffusion\tabularnewline
derivative &  & at fixed point &  & of $\Psi$ &  & and sinks &  & of $\Psi$\tabularnewline
\end{tabular}
\par\end{center}

\mediaMovie[autostart]{\includegraphics[width=0.8\textwidth]{/home/hilary/Videos/pollutionPlumes/pollutionPlumes-still.png}}{animations/Videos_pollutionPlumes_pollutionPlumes.avi}


\section{The Momentum Equation\label{sec:momEqn}}

\begin{center}
\begin{tabular}{ccccccccc}
$\frac{D\mathbf{u}}{Dt}$ & $=$ & $\text{-}2\bm{\Omega}\times\mathbf{u}$ & - & $\frac{\nabla p}{\rho}$ & + & $\mathbf{g}$ & $+$ & $\mu_{u}\left(\nabla^{2}\mathbf{u}+\frac{1}{3}\nabla(\nabla\cdot\mathbf{u})\right)$\tabularnewline
Lagrangian &  & Coriolis &  & Pressure & \multicolumn{3}{c}{Gravitational} & Diffusion\tabularnewline
derivative &  &  &  & gradient & \multicolumn{3}{c}{acceleration} & \tabularnewline
\end{tabular}
\par\end{center}


\subsection{Coriolis}

Inertial Oscillations governed by part of the momentum equation:

\begin{minipage}[c]{0.35\columnwidth}%
\[
\frac{\partial\mathbf{u}}{\partial t}=\text{-}2\bm{\Omega}\times\mathbf{u}
\]

\begin{itemize}
\item \begin{flushleft}
A drifting buoy set in motion by strong westerly winds in the Baltic
Sea in July 1969. 
\par\end{flushleft}
\item \begin{flushleft}
Once the wind subsides, the upper ocean follows inertia circles
\par\end{flushleft}
\end{itemize}
\includegraphics[width=1\linewidth]{figs/copyright_persson_broman}%
\end{minipage}\hfill{}%
\begin{minipage}[c]{0.6\columnwidth}%
\includegraphics[width=1\linewidth]{figs/inertia_circle}%
\end{minipage}


\subsection{The Pressure Gradient Force}

\begin{minipage}[c]{0.48\columnwidth}%
If the pressure gradient force is the only large term in the momentum
equation, then together with the continuity equation and perfect gas
law, we get equations for acoustic waves:
\begin{eqnarray*}
\frac{\partial\mathbf{u}}{\partial t}+\frac{1}{\rho_{0}}\nabla p & = & 0\\
\frac{\partial p}{\partial t}+\rho_{0}c^{2}\nabla\cdot\mathbf{u} & = & 0
\end{eqnarray*}
where $\rho_{0}$ is a reference density and $c$ is the speed of
sound.%
\end{minipage}\hfill{}%
\begin{minipage}[c]{0.48\columnwidth}%
\mediaMovie[autostart,loop]
{\includegraphics[width=\textwidth]{/home/hilary/latex/teaching/MPECDT/PDEsNumerics/notes/anims/Spherical_wave2-0.jpg}}
{animations/latex_teaching_MPECDT_PDEsNumerics_notes_anims_Spherical_wave2.avi}%
\end{minipage}

\clearpage{}


\subsection*{Pressure Gradients lead to very fast acceleration - Acoustic Waves}

\mediaMovie[autostart=true,loop]{\includegraphics[width=0.8\textwidth]{/home/hilary/Videos/pressureWaves/bombBlast.jpg}}{animations/Videos_pressureWaves_bombBlast.avi}

\clearpage{}


\subsection*{Geostrophic Balance: Pressure Gradients versus Coriolis}

\includegraphics[height=0.8\textheight]{figs/synnopticChart}

\clearpage{}


\subsection*{Geostrophic turbulence: pressure gradients, Coriolis and the (non-linear)
advection of velocity by velocity}

\mediaMovie[autostart]{\includegraphics[width=0.8\textwidth]{/home/hilary/Videos/geoTurbulence/geoTurbulence.jpg}}{animations/Videos_geoTurbulence_geoTurbulence.avi}


\subsection{Gravitational Acceleration: A simulation of a dam break}

\noindent \mediaMovie[autostart]{\includegraphics[width=0.8\textwidth]{/home/hilary/Videos/damBreak/damBreak.jpg}}{animations/Videos_damBreak_damBreak.avi}


\subsection*{Gravitational Acceleration: Explosive Comulonimbus}

\noindent \mediaMovie[autostart]{\includegraphics[width=0.8\textwidth]{/home/hilary/Videos/convectivePlumes/explosiveComulonumbus.jpg}}{animations/Videos_convectivePlumes_explosiveComulonumbus.avi}

\clearpage{}


\subsection{Diffusion}

\begin{minipage}[t]{0.47\columnwidth}%
Diffusion of quantity $\phi$ with diffusion coefficient $\mu_{\phi}$
in arbitrary spatial dimensions
\[
\frac{\partial\phi}{\partial t}=\mu_{\phi}\nabla^{2}\phi
\]


And in 1d:
\[
\frac{\partial\phi}{\partial t}=\mu_{\phi}\frac{\partial^{2}\phi}{\partial x^{2}}
\]


\begin{flushleft}
The second derivative of $\phi$ is high at troughs and low in peaks
of $\phi$. Therefore diffusion tends to remove peaks and troughs
and make a profile more smooth:
\par\end{flushleft}


\subsubsection*{Discussion Questions:}
\begin{itemize}
\item Which equations have a diffusion coefficient?
\item What causes diffusion?
\item Is diffusion a large term of the equations of atmospheric motion?\end{itemize}
%
\end{minipage}\hfill{}%
\begin{minipage}[t]{0.49\columnwidth}%
\centering \textbf{Diffusion of a noisy profile}\\
 (zero gradient boundary conditions) \mediaMovie[autostart,loop]
{\includegraphics[width=\textwidth]{/home/hilary/latex/teaching/MTMW12/2014/notes/pythonExamples/diffusion/FTCSdiffusion0.pdf}}
{animations/latex_teaching_MTMW12_2014_notes_pythonExamples_diffusion_FTCSdiffusion.avi}%
\end{minipage}

\clearpage{}


\subsection{The Complete Navier Stokes Equations}

With moisture, phase changes, radiation, ... from NUGAM (courtesy
of Pier Luigi)

\mediaMovie[autostart,loop]
{\includegraphics[width=\textwidth]{/home/hilary/Videos/NUGAM/NUGAM_Sun_1yr_web.jpg}}
{animations/Videos_NUGAM_NUGAM_Sun_1yr_web.avi}
