%% LyX 2.0.6 created this file.  For more info, see http://www.lyx.org/.
%% Do not edit unless you really know what you are doing.
\documentclass[10pt,serif]{beamer}
\usepackage[latin9]{inputenc}
\setcounter{secnumdepth}{3}
\setcounter{tocdepth}{3}
\setlength{\parskip}{0bp}
\setlength{\parindent}{0pt}
\usepackage{url}
\usepackage{bm}
\usepackage{amsmath}
\usepackage{amssymb}
\usepackage{graphicx}
\ifx\hypersetup\undefined
  \AtBeginDocument{%
    \hypersetup{unicode=true,pdfusetitle,
 bookmarks=true,bookmarksnumbered=false,bookmarksopen=false,
 breaklinks=false,pdfborder={0 0 1},backref=section,colorlinks=false}
  }
\else
  \hypersetup{unicode=true,pdfusetitle,
 bookmarks=true,bookmarksnumbered=false,bookmarksopen=false,
 breaklinks=false,pdfborder={0 0 1},backref=section,colorlinks=false}
\fi

\makeatletter

%%%%%%%%%%%%%%%%%%%%%%%%%%%%%% LyX specific LaTeX commands.
%% Because html converters don't know tabularnewline
\providecommand{\tabularnewline}{\\}

%%%%%%%%%%%%%%%%%%%%%%%%%%%%%% Textclass specific LaTeX commands.
 % this default might be overridden by plain title style
 \newcommand\makebeamertitle{\frame{\maketitle}}%
 \AtBeginDocument{
   \let\origtableofcontents=\tableofcontents
   \def\tableofcontents{\@ifnextchar[{\origtableofcontents}{\gobbletableofcontents}}
   \def\gobbletableofcontents#1{\origtableofcontents}
 }
 \long\def\lyxframe#1{\@lyxframe#1\@lyxframestop}%
 \def\@lyxframe{\@ifnextchar<{\@@lyxframe}{\@@lyxframe<*>}}%
 \def\@@lyxframe<#1>{\@ifnextchar[{\@@@lyxframe<#1>}{\@@@lyxframe<#1>[]}}
 \def\@@@lyxframe<#1>[{\@ifnextchar<{\@@@@@lyxframe<#1>[}{\@@@@lyxframe<#1>[<*>][}}
 \def\@@@@@lyxframe<#1>[#2]{\@ifnextchar[{\@@@@lyxframe<#1>[#2]}{\@@@@lyxframe<#1>[#2][]}}
 \long\def\@@@@lyxframe<#1>[#2][#3]#4\@lyxframestop#5\lyxframeend{%
   \frame<#1>[#2][#3]{\frametitle{#4}#5}}
 \def\lyxframeend{} % In case there is a superfluous frame end

%%%%%%%%%%%%%%%%%%%%%%%%%%%%%% User specified LaTeX commands.
\usepackage{animate}

% hack to get natbib working with beamer
\renewcommand{\newblock}{}

% list modifications
\setlength{\leftmargini}{0em}
\setlength{\leftmarginii}{1em}

% make $\times$, $+$, $-$ and $=$ use less space
\newcommand{\tims}{\negthinspace \times \negthinspace}
\newcommand{\eq}{\negthinspace = \negthinspace}
\newcommand{\plus}{\negthinspace + \negthinspace}
\newcommand{\minus}{\text{-}}
\newcommand{\smallcdot}{\negthinspace \cdot \negthinspace}

\newcommand{\nicefrac}[2]{\ensuremath ^{#1}\!\!/\!_{#2}}
\usepackage { fancybox}

\makeatother

\begin{document}

\lyxframeend{}\lyxframe{R-adaptive mesh generation}
\begin{itemize}
\item How to create a mesh which is equidistributed with respect to a monitor
function. ie
\[
A_{i}m_{i}=\text{const}
\]
 for each cell $i$ with area $A_{i}$ for mesh monitor function $m_{i}$. 
\item Define a map from old mesh point, $\bm{\xi}_{i}$, to new mesh points,
$\bm{x}_{i}$ which is the gradient of a potential
\[
\bm{x}=\bm{\xi}+\nabla\phi
\]
This leads to an \textbf{``optimally transported mesh''}, free of
tangling
\item The Jacobian of the mesh transform is the change in area of the cells:
\[
|\nabla\bm{x}|=A_{i}/A_{\xi}
\]

\item The mesh is found by solving a Monge-Amp\'ere equation for $\phi$:
\[
|\nabla\left(\bm{\xi}+\nabla\phi\right)|m\left(x\right)=\text{const}
\]

\item Numerical solution techniques ...
\end{itemize}

\lyxframeend{}


\lyxframeend{}\lyxframe{Spatial Discretisation on the Sphere}
\begin{itemize}
\item Finite Volume Discretisaion on an arbitrary mesh (details in publication
in prep)
\item Exponential maps, to map from one point on a sphere to another point
on a sphere:\textrm{
\[
\mathbf{x}=\bm{\xi}+\exp_{\xi}\nabla\phi
\]
}
\item Geometric interpretation of the Hessian, for cell $i$:
\[
|I+H_{i}\left(\phi\right)|=\frac{A_{i}}{A_{\xi}}
\]
where $A_{i}$ is the area of cell $i$ after the map and $A_{\xi}$
is the area before the map.
\end{itemize}

\lyxframeend{}


\lyxframeend{}\lyxframe{Iterative Solution}

\[
|\nabla\left(\bm{\xi}+\nabla\phi\right)|m\left(x\right)=\text{const}
\]


Use a linearisation of the Hessian term:
\[
|\nabla\left(\bm{\xi}+\nabla\phi\right)|=|I+H\left(\phi\right)|=1+\nabla^{2}\phi+O\left(\phi^{2}\right)
\]
and create fixed-point iterations (Poincar\'e iterations):
\[
1+\nabla^{2}\phi^{\left(n+1\right)}=1+\nabla^{2}\phi^{\left(n\right)}-|I+H\left(\phi^{\left(n\right)}\right)|+\frac{\text{const}^{\left(n\right)}}{m\left(x^{\left(n\right)}\right)}
\]
which can be easily solved to find $\phi^{\left(n+1\right)}$. This
works well if the non-linear terms are small. To maintain stability
for large non-linear terms, solve:
\[
\alpha\nabla^{2}\phi^{\left(n+1\right)}=\alpha\nabla^{2}\phi^{\left(n\right)}-|I+H\left(\phi^{\left(n\right)}\right)|+\frac{\text{const}^{\left(n\right)}}{m\left(x^{\left(n\right)}\right)}
\]
where $\alpha>1$. (Ie increase the size of the linear term that can
be solved implicitly.


\lyxframeend{}


\lyxframeend{}\lyxframe{An Optimally Transported Mesh}

Given the monitor function on the sphere: (latitude-longitude, cylindrical
projection)

\includegraphics[width=1\textwidth]{graphics/monitor}


\lyxframeend{}


\lyxframeend{}\lyxframe{An Optimally Transported Mesh}

The optimally transported mesh is calculated iteratively:

\animategraphics[width=\linewidth,controls,every=2,poster=first]{3}{graphics/meshMovie/mesh}{0}{100}

This is a linear algorithm - cost proportional to number of points.


\lyxframeend{}


\lyxframeend{}\lyxframe{}

\begin{minipage}[c]{0.2\columnwidth}%
\textbf{Discrete Hessian}%
\end{minipage}%
\begin{minipage}[c]{0.78\columnwidth}%
\includegraphics[width=1\linewidth]{graphics/discreteHessianMesh}%
\end{minipage}

\begin{minipage}[c]{0.2\columnwidth}%
\textbf{Geometric Hessian}%
\end{minipage}%
\begin{minipage}[c]{0.78\columnwidth}%
\includegraphics[width=1\linewidth]{graphics/geometricHessianMesh}%
\end{minipage}


\lyxframeend{}


\lyxframeend{}\lyxframe{Convergence to the Monitor Function}

Plots of cell area versus distance to the centre of the refined region 

\begin{tabular}{cc}
\textbf{Discrete Hessian} & \textbf{Geometric Hessian}\tabularnewline
\includegraphics[width=0.49\linewidth]{graphics/discreteHessianArea} & \includegraphics[width=0.49\linewidth]{graphics/geometricHessianArea}\tabularnewline
\end{tabular}


\lyxframeend{}


\lyxframeend{}\lyxframe{Mesh Spacing}

Plots of cell centre to cell centre distance versus distance to the
centre of the refined region 

\begin{tabular}{cc}
\textbf{Discrete Hessian} & \textbf{Geometric Hessian}\tabularnewline
\includegraphics[width=0.49\linewidth]{graphics/discreteHessianDx} & \includegraphics[width=0.49\linewidth]{graphics/geometricHessianDx}\tabularnewline
\end{tabular}


\lyxframeend{}


\lyxframeend{}\lyxframe{Precipitation as a Monitor Function}

\inputencoding{latin1}\vspace{-10pt}


%
\inputencoding{latin9}{\footnotesize{
\[
m=\frac{p+p_{\min}}{p_{\max}+p_{min}}\ \text{where}\ p_{\min}=10^{-5}\text{kg}\text{m}^{-2}\text{s}^{-1},\ p_{\max}=8.73\times10^{-4}\text{kg}\text{m}^{-2}\text{s}^{-1}
\]
}}{\footnotesize \par}

\animategraphics[width=\linewidth,controls,poster=first]{100}{graphics/pptMonitorMovie/pptMonitor}{0}{11}

%
{\small{Using daily average precipitation rate, 1-12 Oct 2012, from
the NOAA-CIRES 20th Century Reanalysis version 2 (Compo et al, 2011,
\url{http://www.esrl.noaa.gov/psd/data/gridded/data.20thC_ReanV2.html})\vfill{}
}}{\small \par}

%

\lyxframeend{}


\lyxframeend{}\lyxframe{Conclusions}
\begin{itemize}
\item Exponential maps used to create optimally transported meshes on the
sphere
\item New linearisation of the Hessian to create a semi-implicit algorithm
\item Geometric version of the Hessian gives accurately equidistribued meshes
\item Discrete version of the Hessian gives some extra large and some extra
small cells
\item Geometric Hessian gives less anisotropy
\end{itemize}

\lyxframeend{}
\end{document}
